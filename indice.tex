\begingroup % un grupo para que o de parindent non afecte ao resto da revista
\setlength\parindent{0pt}
% Color del resalte trasparentado a la derecha del índice
\begin{tikzpicture}[remember picture, overlay]
    \draw[
        line width = 0.73\textwidth,
        color      = Resalte!65!white
    ] ($(current page.north east) - (0,0)$) -- ($(current page.south east) - (0,0)$);
\end{tikzpicture}
% (non xoguedes a poñer liñas en blanco entre as tikzpictures, vai ir mal)
%
% %%%%%%%%%%%%%%%%%%%%%%%%%%%%%%%%%%%%%%%%%%%%%%%%%%%%%%%%%%%%%%%%%%%%%%%%%%%%
% Columna de la izqueirda, está el índice y el Sobre Momentum. Para colocar ben
% as as cousas hai que usar moitos 'minipage'.
\begin{minipage}[t]{0.72\textwidth}%
    \vspace{-0.3cm}%
    \begin{minipage}[t]{\hsize}%
        \makebox[0pt][l]{%
            % :FACER: se metemos moitas cousas, non colle
            %
            % ============ TOC ===============================================
            % Aquí incluimos el tableofcontens. Personalizado aparte no arquivo
            % 'revista.cls', botádelle un ollo ao comando \SeccionTOC
            \begin{adjustbox}{valign=t,minipage={0.9\textwidth},margin={0pt,0pt}}
                {
                    \section*{\Huge \textbf{Índice}}
                    \begingroup%
                        % Esto es para que no salte el Indice default, y lo
                        % podamos poner nosotros
                        \let\contentsname\relax%
                        \tableofcontents% <--- o 'TOC' CARGASE AQUí
                    \endgroup%
                }%
            \end{adjustbox}%
        }%
        % ============== SOBRE MOMENTUM =======================
        % Texto debajo del índice, sobre momentum
        \vfill
        \begin{minipage}[t]{0.9\textwidth}
            \vfill
            \vspace{2cm}
            %{\Huge \color{Resalte!75!black} \textbf{Sobre Momentum}:} \\[1cm]
            %\imprimeSobreMomentum
        \end{minipage}
    \end{minipage}
\end{minipage}
\hfill
%
% %%%%%%%%%%%%%%%%%%%%%%%%%%%%%%%%%%%%%%%%%%%%%%%%%%%%%%%%%%%%%%%%%%%%%%%%%%%%
% %%%%%%%%%%%%%%%%%%%%%%%%%%%%%%%%%%%%%%%%%%%%%%%%%%%%%%%%%%%%%%%%%%%%%%%%%%%%
% LADO DEREITO COS PARTICIPANTES, VARIOS LINKS, DATA, etc. Resaltado cunha cor
%
\begin{minipage}[t]{0.28\textwidth}
    \vspace{-0.3cm}
    \hspace{-0.7cm}
    % Data e Número da revista
    \begin{center}
        \textcolor{TextoEnResalte}{\Large
            \textbf{\today }\\[5mm]
            \textbf{\imprimeNumero} \\[1.5cm]
        }
    \end{center}
    % Aquí mostramos as persoas que participaron no proxecto.
    \imprimeParticipantes \\[1.5cm]
    % Links varios
    \begin{tikzpicture}[scale=1.5]
        \begingroup
        \hypersetup{urlcolor = TextoEnResalte}
        \node at (0,0) { \FonteSimbolos\fontsize{20pt}{0pt}\selectfont \textcolor{TextoEnResalte}  };
        \node[
            anchor = west,
            align  = left
        ] at (0.3,0)
        {
            \footnotesize \href{mailto:\imprimeCorreo}{\textbf{\imprimeCorreo}}
        };
        % \node at (0,-1) {\FonteSimbolos\fontsize{20pt}{0pt}\selectfont \textcolor{TextoEnResalte}  };
        % \node[
        %     anchor = west,
        %     align  = left
        % ] at (0.5,-1)
        % {
        %     \href{https://github.com/\imprimeLinkRepositorio}{\texttt{\imprimeLinkRepositorio}}
        % };
        \endgroup
    \end{tikzpicture} \\[1.7cm]
    %
    % Logo da USC
    \begin{center}
        \includegraphics[width=0.8\linewidth]{logos/usc-negativo-escuro.pdf}
    \end{center}
\end{minipage}
\endgroup % <-- este cerra o grupo que contén o de \setlenght\parindent{0pt}
\newpage
\pagenumbering{arabic}
