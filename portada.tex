\pagenumbering{gobble}
\thispagestyle{empty}

% Creamos unha xeometria nova
\newgeometry{
    top    = 2mm,  % marxe superior
    left   = 5mm,  % marxe esquerdo
    right  = 5mm,  % marxe dereito
    bottom = 5mm,  % marxe inferior
    nohead = true, % desactivar o encabezado
    nofoot = true, % desactivar o pe de paxina
}

\begingroup % un grupo para que o de parindent non afecte ao resto da revista
\setlength\parindent{0pt}

% TITULO %%%%%%%%%%%%%%%%%%%%%%%%%%%%%%%%%%%%%%%%%%%%%%%%%%%%%%%%%%%%%%%%%%%%%
\vspace*{3mm}
{%
    \centering%
    \fontsize{70pt}{0pt}\selectfont%
    \FormateaTitulo{Resalte}\par%
}%
\vspace*{6mm}
{%
    \centering%
    \fontsize{18pt}{0pt}\selectfont%
    \textsf{A revista estudantil da Facultade de Física da USC}\par%
}%
\vspace*{7mm}

% NUMERO DA REVISTA E DATA %%%%%%%%%%%%%%%%%%%%%%%%%%%%%%%%%%%%%%%%%%%%%%%%%%%
{
\setstretch{0}
\begin{tcolorbox}[
    colback  = Resalte,
    colframe = black,
    boxrule  = 2pt,
    boxsep   = 6pt,
    sharp corners,
]%
    {
        \fontsize{20pt}{0pt}\selectfont%
        \color{TextoEnResalte}\texttt{\imprimeNumero\hfill\imprimeData}\par%
    }%
\end{tcolorbox}
}

% IMAXE DE CADA REVISTA %%%%%%%%%%%%%%%%%%%%%%%%%%%%%%%%%%%%%%%%%%%%%%%%%%%%%%
% A imaxe da portada debe ser CADRADA EXACTAMENTE
{
\vspace*{-4pt}
\nointerlineskip
\begin{picture}(\textwidth,\textwidth)%
    \put(0,0){%
        \setlength{\fboxsep}{0pt}%
        \setlength{\fboxrule}{2pt}%
        \fbox{%
            \includegraphics[width=\textwidth-4pt]{\imprimeImaxePortada}%
        }%
    }%
\end{picture}
}

% IMAXES SUPERPOSTAS NA IMAXE DA PORTADA %%%%%%%%%%%%%%%%%%%%%%%%%%%%%%%%%%%%%
\begin{tikzpicture}[remember picture, overlay]
    % Imagen 1 (superpuesta)
    \node[
        anchor       = south west,
        xshift       = 0.9cm,
        yshift       = 3.7cm,
        fill         = gray,
        fill opacity = 0.75, % transparencia del fondo
        text opacity = 1,    % el texto sigue opaco
        % draw,              % opcional, para dibujar el borde
        % rounded corners,   % opcional, para esquinas redondeadas
        inner sep=5pt        % separación interna del texto al borde
    ]  at (current page.south west)
    {\normalsize \textcolor{white}{\imprimeComentarioImaxePortada} };

    % Imagen 2 (superpuesta)
    \node[
        anchor = south west,
        xshift = 0.55cm,
        yshift = 0.5cm
    ] at (current page.south west)
    {\includegraphics[width=12cm]{logos/vicerreitoria-branco-negro.pdf}};
\end{tikzpicture}
\endgroup% <-- este cerra o grupo que contén o de \setlenght\parindent{0pt}

% volvemos a cargar a xeometria que tiñamos no preambulo
\restoregeometry
