\newgeometry{
    top    = 5mm,  % marxe superior
    left   = 6mm,  % marxe esquerdo
    right  = 6mm,  % marxe dereito
    bottom = 8mm,  % marxe inferior
    nohead = true, % desactivar o encabezado
    nofoot = true, % desactivar o pe de paxina
}

% FONDO %%%%%%%%%%%%%%%%%%%%%%%%%%%%%%%%%%%%%%%%%%%%%%%%%%%%%%%%%%%%%%%%%%%%%%
\AddToShipoutPictureBG*{
    \put(0,-10){
        \parbox[b][1.2\paperheight]{\paperwidth}{
            \vfill
            \centering
            {\transparent{0.05}\includegraphics{logos/botafumeiro.png}}
            \vfill
        }
    }
}
% No caso de poñelo en todas as páxinas, pode pararse con \ClearShipoutPicture
%
% :FACER: imos cambiar este texto en cada revista?
%
{% os textos de agradecemento e similares
\vspace*{2.4cm}
\begin{center}
    \begin{minipage}{0.68\linewidth}
%    \centering
    {\imprimeDespedida }
\end{minipage} \\[1.5cm]

% Estes textos SEGURO que se cambian en cada revista. Hai que metelo nun macro
\begin{minipage}{0.75\linewidth}
    \centering
    { \imprimeAgradecementos }
    \end{minipage}
\end{center}
}

\thispagestyle{empty}
\vspace*{3em}

\vfill
\vspace{-1cm}
\hrulefill

% :FACER: tal vez facer variables cos contidos das URLs do QR e dos contactos..?

% O do QR non sei por qué pero a veces colocase mal. Mirando con
% lua-visual-debug pareceme que crea un numero incrible de boxes que pode que
% toleen a colocacion doutras cousas. Por agora furrula, polos pelos.
\begin{minipage}[c][5cm]{5cm}
        \begin{center}
            Edicións anteriores: \\[2mm]
        \hypersetup{urlcolor=black}
        \qrset{height = 4cm}%
        % Que quede para a posteridade que rickrolleei a Celia e a Sebas con este QR
        \qrcode{\imprimeEdicionsAnteriores}
    \end{center}
\end{minipage}
%
\hfill % non se deben poñer liñas en branco arredor deste \hfill
%
\begin{minipage}[c][5cm]{4cm}
    \begin{center}
        Participa! (WhatsApp) \\[2mm]
        \hypersetup{urlcolor=black}
        \qrset{height = 4cm}%
        % Que quede para a posteridade que rickrolleei a Celia e a Sebas con este QR
        \qrcode{\imprimeWhatsApp}
    \end{center}
\end{minipage}
%
%
\hfill % deste tampouco
%
% :FACER: pasar .eps -> .pdf leva tempo. Igual deberíamos usar o PDF
% directamente
%
% Sobre o logo en formato .eps. O orixinal ven de:
% https://nubeusc.sharepoint.com/sites/servizos-oficina-web/Documentos%20compartidos/Forms/AllItems.aspx?csf=1&web=1&FolderCTID=0x012000441AD9196B55D84292BD3BC4FC87F798&id=%2Fsites%2Fservizos%2Doficina%2Dweb%2FDocumentos%20compartidos%2FImaxe%20corporativa%2FLogotipo%20da%20USC&viewid=a5e177b5%2D7018%2D46d6%2D9352%2Dfc57158bf6b7
% (espero que a ligazón dure) O arquivo .eps debe pasarse a PDF con 'epstopdf'.
% Esto debería facerse automáticamente, supoñendo que existe o executable.
% Prefiro facelo así porque o que comparten da USC ten ese formato
% Co logo da vicerreitoría
\begin{minipage}[c][4cm]{9cm}
\begin{center}
   Co financiamento de: \\[2mm]
    \begin{picture}(5cm,4cm)%
        \put(-60,30){\hbox{\includegraphics[width=11cm]{logos/vicerreitoria-branco-negro.pdf}}}
    \end{picture}
\end{center}
\end{minipage}
%
% Co logo da USC
%\begin{minipage}[c][4cm]{5cm}
%    \begin{picture}(5cm,4cm)%
%        \put(0,0){\hbox{\includegraphics[width=5cm]{logos/usc-branco-negro.eps}}}
%    \end{picture}
%\end{minipage}