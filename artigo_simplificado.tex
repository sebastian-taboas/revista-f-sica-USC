%%%%%%%%%%%%%%%%%%%%%%%%%%%%%%%%%%%%%%%%%%%%%%%%%%%%%%%%%%%%%%%%%%%%%%%%%%%%%%
% Este é o modelo dun artigo simplificado. NON ESTÁ COMPLETO. ESTO NO É A
% REVISTA FINAL. A revista final componse de varios arquivos que inclúen o que
% aquí é o corpo do documento (dentro do \begin{document}). Esto so sirve para
% poder compilar un artigo individual facilmente, pero non inclue a portada,
% contraportada, e varios macros non están definidos. Esto no meu sistema
% compílase nuns 10 s. A revista_001 completa, en case 30 s
%
% Para usalo fai falta poñer 3 cousas no mesmo diretorio
% a clase da revista:   revista.cls
% este artigo:          modelo_artigo.tex
% as tipografias:       fontes\
%
% Opcionalmente, tamén é util ter o arquivo de configuracion 'latexmkrc', pero
% non é necesario.
%
% COMPILACION:
% Necesítase LuaLaTeX, non funciona doutro modo (nin o imos cambiar). Tamén se
% precisa Biber para a bibliografía
%
% Na maioría de editores debería bastar con premer calquer botón de 'executar'.
% Se non funciona seguramente teñades que cambiar o método que usa voso editor
% detrás de cámaras. Recordo que TeXStudio tiña un panel que deixaba
% seleccionar qué compiladores/executables queremos usar. O mellor é usar
% 'latexmk', o cal vai, á sua vez, elixir qué compiladores/executables se usan
% de xeito automático
%%%%%%%%%%%%%%%%%%%%%%%%%%%%%%%%%%%%%%%%%%%%%%%%%%%%%%%%%%%%%%%%%%%%%%%%%%%%%%

% Esto importa a clase definida no arquivo aparte, 'revista.cls'
\documentclass{revista}

\begin{document}

% Este macro hai que usalo sempre. Fai bastantes cousas, como engadir o titulo
% e autor ao índice, resetear os conteos de figuras, e evidentemente imprimir
% no PDF a propia información de cada artigo.
%
% \Titular   -> sen asterisco, non engade a sección ao indice
% \Titular*  -> con asterisco, engade a sección ao indice.
%
% Ao ser unha versión simplificada da revista, da igual cal usedes. Eso xa se
% cambiará logo na propia edición
%
% {TITULO}   Evidente, o titulo do artigo. Non debería ser moi longo.
% {PERSONA}  Autores do artigo
% {TEMA}     A escoller entre:
%               divulgacion,historia,actualidadeFacultade,
%               actualidadeCientifica,filosofia,entrevistas,
%               programacion,pasatempos,anuncios,profesorado
% {SUBTITULO} Un pequeno texto algo máis explicativo (e posiblemente longo) co
%             titulo. E o único argumento do titular que pode quedar en branco
\Titular%
{TITULO}%
{PERSONA}%
{divulgacion}%
{SUBTITULO}%


% 'refsection' sirve para que ao final, ao mostrar a bibliografía, esta solo
% inclua as entradas que foron citadas DENTRO desta 'refsection' con \cite. É
% opcional, a menos que usades bibliografía, entón é obligatorio
\begin{refsection}

% O CORPO DO ARTIGO COMEZA DE AQUÍ EN DIANTE
%%%%%%%%%%%%%%%%%%%%%%%%%%%%%%%%%%%%%%%%%%%%%%%%%%%%%%%%%%%%%%%%%%%%%%%%%%%%%%
%vvvvvvvvvvvvvvvvvvvvvvvvvvvvvvvvvvvvvvvvvvvvvvvvvvvvvvvvvvvvvvvvvvvvvvvvvvvvv

% 'multicols' danos bastante flexibilidade para colocar as cousas en duas
% columnas. Se unha parte non debe ter 2 columnas, solo hai que rematar o
% entorno 'multicols' antes, escribir o que queiramos (qedará en 1 columna), e
% volver iniciar outro entorno 'multicols' para o resto
\begin{multicols}{2}


% Os únicos "niveis" que se permiten son de 'subsection' en abaixo. En
% principio poden estar numerados ou no, a gusto de cada un
\subsection*{Introdución}

Texto texto texto texto texto texto texto texto texto texto texto texto texto
texto texto texto texto texto texto texto texto

\subsection*{Introdución}

Texto texto texto texto texto texto texto texto texto texto texto texto texto
texto texto texto texto texto texto texto texto

% Finalmente mostrar a bibliografía. Recordo que é importantísimo que esto esté
% dentro do entorno 'refsection'. Engadir que a bibliograia manéxase con
% Biblatex
\printbibliography

\end{multicols}
\end{refsection}

\end{document}
