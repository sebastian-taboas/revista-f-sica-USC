\documentclass{revista}

\Numero{001}
\Data{Abril 2025}
\ImaxePortada{./revistas/001/imaxes/pedra.jpg}
\ComentarioImaxePortada{\textbf{1981: Primeira pedra da facultade de física.}}
\CorResalte{ff0000}
\CorTextoEnResalte{ffffff}
\SobreMomentum{
    Tras uns cantos meses de traballo, por fin podemos dar saída á nova revista
    estudantil \textit{Momentum}, unha revista que busca ser un medio de
    comunicación tanto dentro coma fóra da facultade de física onde expor os
    intereses científicos do estudantado. Cuestións de Divulgación, Actualidade,
    Entrevistas, Historia, Filosofía da Ciencia, Opinión, Programación... Son todos
    temas que teñen cabida dentro deste proxecto. Esta revista está realizada
    integramente polo estudantado da Facultade de Física USC, onde pretendemos ter
    un recuncho de expresión máis aló do estrictamente académico. Fundada no ano
    2025 co obxectivo de persistir na historia, esforzámonos sempre en mellorar.
    Non dubidedes en deixar a vosa pegada!
}
\Participantes{%
    {\Large \textbf{Dirección}}     \\[0.5cm] %
        Álvaro Pallas Otero          \\[0.2cm] %
        Sebastián Táboas Pazo        \\[0.25cm]% 
        Celia Álvarez Álvarez        \\[0.25cm]%
        Daniel Vázquez Lago          \\[0.75cm]%
    {\Large \textbf{Edición}}        \\[0.5cm]%
        David Cotelo Varela          \\[0.2cm]%
        Víctor Díaz Díaz             \\[0.2cm]%
        Daniel Vázquez Lago          \\[0.2cm]%
        Manuel Vázquez Carreira      \\[0.25cm]%
        Ana Díaz Caride              \\[0.25cm]%
        Cristóbal Santos Sánchez     \\[0.2cm] %
        Mauro Garrido Rodríguez      \\[0.70cm]%
    {\Large \textbf{Deseño de Logo}} \\[0.5cm]%
        Ana Díaz Caride              \\[0.2cm]%
}
\Despedida{
    Momentum! \\[6mm]
    Aquí chega a revista por e para estudantes da Facultade de Física USC!
    Cansos de que o momento lineal e angular guíen as nosas traxectorias, imos
    escribir unha nova historia; entrevistas, divulgación, filosofía da ciencia e
    moitos artigos dispares cargamos coa inercia de formar unha nova fiestra para o
    alumnado. Tes nas túas mans esta oportunidade, deixa que o magnetismo te leve e
    participa, sé parte deste proxecto: escribe, le, comparte, suxestiona… A
    revista é real e as túas ideas poden ser máis que imaxinación, non dubides en
    deixar a túa pegada neste recuncho físico, onde hai física máis aló das aulas.
    Non dubides unirte ao proxecto!
}
\Agradecementos{
    Agradecementos: \\[5mm]
    Dende a dirección da revista, queríamos agradecervos a todos por achegarvos
    a este proxecto. Non hai revista sen lector! Pero, para facela, estivo moita
    xente implicada, que non podemos pasar por alto. \\[5mm]
    Gracias ao noso estimado J. A. Docobo pola súa predisposición e boa vontade
    para participar nun proxecto que aínda non existía, e a Ánxel Costas por
    poñernos en contacto. \\[5mm]
    Gracias a todas as persoas que se interesaron polo proxecto, simplemente
    por intereresarse, aínda que ao final non participaran directamente. \\[5mm]
    E por suposto, tamén temos que mencionar ao equipo decanal da nosa
    facultade, que nos apoiou e motivou dende o primeiro momento, e ofreceu
    financiar a impresión e colaborar con nós nesta primeira edición. Non sería
    posible sen eles!
}

\begin{document}
\pagenumbering{gobble}
\thispagestyle{empty}

% Creamos unha xeometria nova
\newgeometry{
    top    = 2mm,  % marxe superior
    left   = 5mm,  % marxe esquerdo
    right  = 5mm,  % marxe dereito
    bottom = 5mm,  % marxe inferior
    nohead = true, % desactivar o encabezado
    nofoot = true, % desactivar o pe de paxina
}

\begingroup % un grupo para que o de parindent non afecte ao resto da revista
\setlength\parindent{0pt}

% TITULO %%%%%%%%%%%%%%%%%%%%%%%%%%%%%%%%%%%%%%%%%%%%%%%%%%%%%%%%%%%%%%%%%%%%%
\vspace*{3mm}
{%
    \centering%
    \fontsize{70pt}{0pt}\selectfont%
    \FormateaTitulo{Resalte}\par%
}%
\vspace*{6mm}
{%
    \centering%
    \fontsize{18pt}{0pt}\selectfont%
    \textsf{A revista estudantil da Facultade de Física da USC}\par%
}%
\vspace*{7mm}

% NUMERO DA REVISTA E DATA %%%%%%%%%%%%%%%%%%%%%%%%%%%%%%%%%%%%%%%%%%%%%%%%%%%
{
\setstretch{0}
\begin{tcolorbox}[
    colback  = Resalte,
    colframe = black,
    boxrule  = 2pt,
    boxsep   = 6pt,
    sharp corners,
]%
    {
        \fontsize{20pt}{0pt}\selectfont%
        \color{TextoEnResalte}\texttt{\imprimeNumero\hfill\imprimeData}\par%
    }%
\end{tcolorbox}
}

% IMAXE DE CADA REVISTA %%%%%%%%%%%%%%%%%%%%%%%%%%%%%%%%%%%%%%%%%%%%%%%%%%%%%%
% A imaxe da portada debe ser CADRADA EXACTAMENTE
{
\vspace*{-4pt}
\nointerlineskip
\begin{picture}(\textwidth,\textwidth)%
    \put(0,0){%
        \setlength{\fboxsep}{0pt}%
        \setlength{\fboxrule}{2pt}%
        \fbox{%
            \includegraphics[width=\textwidth-4pt]{\imprimeImaxePortada}%
        }%
    }%
\end{picture}
}

% IMAXES SUPERPOSTAS NA IMAXE DA PORTADA %%%%%%%%%%%%%%%%%%%%%%%%%%%%%%%%%%%%%
\begin{tikzpicture}[remember picture, overlay]
    % Imagen 1 (superpuesta)
    \node[
        anchor       = south west,
        xshift       = 0.9cm,
        yshift       = 3.7cm,
        fill         = gray,
        fill opacity = 0.75, % transparencia del fondo
        text opacity = 1,    % el texto sigue opaco
        % draw,              % opcional, para dibujar el borde
        % rounded corners,   % opcional, para esquinas redondeadas
        inner sep=5pt        % separación interna del texto al borde
    ]  at (current page.south west)
    {\normalsize \textcolor{white}{\imprimeComentarioImaxePortada} };

    % Imagen 2 (superpuesta)
    \node[
        anchor = south west,
        xshift = 0.55cm,
        yshift = 0.5cm
    ] at (current page.south west)
    {\includegraphics[width=12cm]{logos/vicerreitoria-branco-negro.pdf}};
\end{tikzpicture}
\endgroup% <-- este cerra o grupo que contén o de \setlenght\parindent{0pt}

% volvemos a cargar a xeometria que tiñamos no preambulo
\restoregeometry

\begingroup % un grupo para que o de parindent non afecte ao resto da revista
\setlength\parindent{0pt}
% Color del resalte trasparentado a la derecha del índice
\begin{tikzpicture}[remember picture, overlay]
    \draw[
        line width = 0.73\textwidth,
        color      = Resalte!65!white
    ] ($(current page.north east) - (0,0)$) -- ($(current page.south east) - (0,0)$);
\end{tikzpicture}
% (non xoguedes a poñer liñas en blanco entre as tikzpictures, vai ir mal)
%
% %%%%%%%%%%%%%%%%%%%%%%%%%%%%%%%%%%%%%%%%%%%%%%%%%%%%%%%%%%%%%%%%%%%%%%%%%%%%
% Columna de la izqueirda, está el índice y el Sobre Momentum. Para colocar ben
% as as cousas hai que usar moitos 'minipage'.
\begin{minipage}[t]{0.72\textwidth}%
    \vspace{-0.3cm}%
    \begin{minipage}[t]{\hsize}%
        \makebox[0pt][l]{%
            % :FACER: se metemos moitas cousas, non colle
            %
            % ============ TOC ===============================================
            % Aquí incluimos el tableofcontens. Personalizado aparte no arquivo
            % 'revista.cls', botádelle un ollo ao comando \SeccionTOC
            \begin{adjustbox}{valign=t,minipage={0.9\textwidth},margin={0pt,0pt}}
                {
                    \section*{\Huge \textbf{Índice}}
                    \begingroup%
                        % Esto es para que no salte el Indice default, y lo
                        % podamos poner nosotros
                        \let\contentsname\relax%
                        \tableofcontents% <--- o 'TOC' CARGASE AQUí
                    \endgroup%
                }%
            \end{adjustbox}%
        }%
        % ============== SOBRE MOMENTUM =======================
        % Texto debajo del índice, sobre momentum
        \vfill
        \begin{minipage}[t]{0.9\textwidth}
            \vfill
            \vspace{2cm}
            %{\Huge \color{Resalte!75!black} \textbf{Sobre Momentum}:} \\[1cm]
            %\imprimeSobreMomentum
        \end{minipage}
    \end{minipage}
\end{minipage}
\hfill
%
% %%%%%%%%%%%%%%%%%%%%%%%%%%%%%%%%%%%%%%%%%%%%%%%%%%%%%%%%%%%%%%%%%%%%%%%%%%%%
% %%%%%%%%%%%%%%%%%%%%%%%%%%%%%%%%%%%%%%%%%%%%%%%%%%%%%%%%%%%%%%%%%%%%%%%%%%%%
% LADO DEREITO COS PARTICIPANTES, VARIOS LINKS, DATA, etc. Resaltado cunha cor
%
\begin{minipage}[t]{0.28\textwidth}
    \vspace{-0.3cm}
    \hspace{-0.7cm}
    % Data e Número da revista
    \begin{center}
        \textcolor{TextoEnResalte}{\Large
            \textbf{\today }\\[5mm]
            \textbf{\imprimeNumero} \\[1.5cm]
        }
    \end{center}
    % Aquí mostramos as persoas que participaron no proxecto.
    \imprimeParticipantes \\[1.5cm]
    % Links varios
    \begin{tikzpicture}[scale=1.5]
        \begingroup
        \hypersetup{urlcolor = TextoEnResalte}
        \node at (0,0) { \FonteSimbolos\fontsize{20pt}{0pt}\selectfont \textcolor{TextoEnResalte}  };
        \node[
            anchor = west,
            align  = left
        ] at (0.3,0)
        {
            \footnotesize \href{mailto:\imprimeCorreo}{\textbf{\imprimeCorreo}}
        };
        % \node at (0,-1) {\FonteSimbolos\fontsize{20pt}{0pt}\selectfont \textcolor{TextoEnResalte}  };
        % \node[
        %     anchor = west,
        %     align  = left
        % ] at (0.5,-1)
        % {
        %     \href{https://github.com/\imprimeLinkRepositorio}{\texttt{\imprimeLinkRepositorio}}
        % };
        \endgroup
    \end{tikzpicture} \\[1.7cm]
    %
    % Logo da USC
    \begin{center}
        \includegraphics[width=0.8\linewidth]{logos/usc-negativo-escuro.pdf}
    \end{center}
\end{minipage}
\endgroup % <-- este cerra o grupo que contén o de \setlenght\parindent{0pt}
\newpage
\pagenumbering{arabic}


\setlength{\parskip}{1.8mm} % Cambia el espacio entre párrafos

\Titular*%
{Benvida a \textit{MOMENTUM}}%
{Equipo Decanal}%
{profesorado}%
{}%

\vspace{-1.1cm}

\begin{multicols}{2}

Dende o Decanato da Facultade saudamos con entusiasmo a iniciativa de abrir un
novo espazo onde o estudantado poida plasmar as súas inquedanzas e intereses ao
redor da súa estadía na nosa Facultade de Física da USC. A revista
\textit{Momentum} recolle o testemuño de anteriores iniciativas similares que
existiron na Facultade, nomeadamente as revistas A \textit{Gota de Millikan} e
o \textit{Físical Interviú} que gozaron de moito éxito entre os colectivos da
Facultade. Comparten compromiso coa divulgación, xestión e participación
estudiantil e diversidade de contidos. As tres publicacións, ao igual que o QMF
ou a Quantum Cup reactivada este ano, constitúen unha ponte que conecta varias
xeracións de físicos e físicas separadas por case corenta anos. Da súa lectura
se percebe unha esencia común que perdura adaptada á evolución do contexto
socio-económico e científico-tecnolóxico.

A \textit{Gota de Millikan} naceu en 1988, pouco despois que a Facultade, cunha
orientación fundamentalmente divulgadora promovida pola Asociación Isaac
Newton. Publicaban traballos moi rigorosos, asinados tanto polo estudantado
como polo profesorado, que trataban diversas temáticas relacionadas coas
ciencias experimentais e matemáticas. Entre o estudantado asinante podedes
atopar profesorado de sobra coñecido por vós, algún xa xubilado. A \textit{Gota
de Millikan} adicaba unha sección ao xadrez, unha das actividades favoritas do
estudantado de Física da época. Eran míticas as análises das xogadas de Carril
que poñía en práctica na cafetaría de Pachín situada no espazo que ocupa
actualmente a Sala de Lectura. Esta tradición mantívose e consolidouse co tempo
coa celebración periódica de torneos de xadrez na Facultade de Física da man de
Suso Mosqueira que atraen a toda a comunidade universitaria. Incorporaba tamén
unha sección de Paradoxas que invitaba aos lectores a pensar. Para sufragar o
custo que supoñía ser editada en papel, incorporaba anuncios publicitarios de
establecementos vinculados á actividade diurna e nocturna do estudantado que
hoxe espertan un sentimento doce de nostalxia. Publicáronse 9 números entre
1988 e 1992, e un décimo e derradeiro en 1997 que podedes atopar na Biblioteca.
Existiu nunha época na que o acceso a un ordenador persoal non era tan común e
moito menos o acceso ao correo electrónico; nunha época na que as redes sociais
estaban constituídas por esa lista de coñecidos cos que interactuabas na
universidade, no transporte, á hora de xantar en comedores ou bares, nos
locutorios, no tempo de ocio...; rede que se ía ampliando con cada apretón de
mans ou par de bicos tras unha presentación e que se ía consolidando co
contacto periódico pois nin o teléfono fixo estaba ao alcance de todos.

\textit{Físical Interviú} aproximábase máis a un boletín. O seu formato era
máis flexible (entrevistas, opinións, divulgación científica, novas, cartas,
crónica social, crónica política...) e os seus contidos, máis variados, foron
evolucionando co tempo. Os avances nas comunicacións facilitaron o acceso á
información e a persoal científico ou emprendedor alleo á USC que contribuiu
con colaboracións puntuais. En \textit{Físical Interviú} tratábanse temas
relacionados coa vida na Facultade: o edificio, a docencia, a investigación ou
os logros no deporte por parte do estudantado, por exemplo. Constitúe un
verdadeiro documento gráfico que lembra acontecementos e momentos  importantes
como os encontros do alumnado cos Premios Nobel, grazas ao programa ConCiencia;
as distintas edicións do QMF; ou o renacer da Asociación Isaac Newton; a
constitución da Delegación do alumnado, a adaptación ao Espazo Europeo de
Educación Superior. Foi un medio pioneiro na visibilización da muller na Física
co artigo \textit{Mi mamá es investigadora} no ano 2008 e a sección ConCiencia
de Muller nos últimos tempos.. Estivo activa de xeito intermitente ata 2021, un
período moi extenso suxeito a moitos cambios. Os avances en tecnoloxía dixital
e informática permitiron co tempo publicar edicións dixitais, que estaban
dispoñibles nun espazo web, ademais das tiradas impresas. Algúns dos exemplares
impresos  publicados tamén están dispoñibles na Biblioteca para consulta.

O contexto no que xorde \textit{Momentum} é distinto ao doutras épocas, pois as
formas de comunicación evolucionaron de maneira significativa. Hoxe, o espazo
público do debate e da reflexión está amplamente dominado polas redes sociais,
que permiten a interacción inmediata e a difusión rápida de contidos. Porén,
estas plataformas teñen tamén as súas limitacións: a inmediatez moitas veces
non deixa espazo para a reflexión pausada, para a análise en profundidade ou
para o intercambio estruturado de ideas. \textit{Momentum}, pola súa natureza
como publicación periódica, pode suplir esa carencia ao ofrecer un lugar onde
as ideas se expresen con maior desenvolvemento, onde se poidan articular
argumentos sólidos e onde se fomente un diálogo enriquecedor sobre os temas que
máis preocupan e interesan ao estudantado e ao profesorado.

A nosa experiencia dinos que o éxito de \textit{Momentum} será directamente
proporcional á implicación que teña o estudantado á hora de participar nela.
Unha publicación deste tipo pode converterse nun valioso engadido ao conxunto
de espazos de expresión dentro da Facultade. Non só permitirá compartir
coñecementos e experiencias, senón que tamén servirá como rexistro histórico do
que acontece en cada etapa. En anos futuros, as novas xeracións de estudantes
poderán botar unha ollada ás páxinas de \textit{Momentum} e descubrir como eran
os debates, as preocupacións e as ilusións da comunidade universitaria do seu
tempo.

Seguro que abundarán as referencias ás distintas actividades que teñen lugar na
Facultade. Temos a sorte de contar cun estudantado moi activo, que enche o
curso académico con numerosas iniciativas: ceas, concertos como o sacrosanto
Quantum Music Festival (QMF), actos de graduación, actividades culturais e
deportivas, entre outras. Todas estas accións contribúen a crear un ambiente
universitario vivo e dinámico, no que non todo se reduce ás aulas e aos
laboratorios. \textit{Momentum} pode ser o espazo idóneo para recoller estas
experiencias, para narrar as anécdotas que xorden delas e para reflectir a
diversidade de intereses do noso estudantado.

Pero non só sabemos de festa. Desde a Facultade tamén se promoven moitas outras
iniciativas de carácter académico e científico, sendo as máis relevantes os
diversos congresos, cursos e seminarios que teñen lugar ao longo do ano.
Ademais, levamos anos organizando a fase galega da Olimpiada de Física, en
coordinación coa Real Sociedade Española de Física, ofrecendo aos estudantes de
secundaria a oportunidade de achegarse ao mundo da Física con retos e desafíos
intelectuais estimulantes. Unha publicación como \textit{Momentum} pode servir
para documentar e destacar estas actividades, dándolles visibilidade e
recoñecendo o esforzo de quen as fai posibles.

Moitas veces, as cousas que suceden na Facultade pasan desapercibidas porque
estamos inmersos na rutina diaria e non sempre somos conscientes da riqueza da
nosa contorna. \textit{Momentum} pode axudar a cambiar iso, ofrecendo un
escaparate para dar a coñecer proxectos, experiencias e historias que, doutro
xeito, poderían quedar relegadas ao esquecemento. Unha revista feita polo
estudantado e para o estudantado ten o potencial de reflectir a verdadeira
esencia da Facultade de Física: a súa diversidade, a súa vitalidade e o seu
espírito crítico e innovador.

Ademais, \textit{Momentum} pode servir como vehículo para a divulgación
científica dentro da Facultade. Contamos con expertos divulgadores recoñecidos
a nivel nacional e internacional polo seu xeito sinxelo e efectivo de explicar
a Ciencia á poboación xeral. O feito de que o noso estudantado tamén estea
implicado en iniciativas de divulgación, como o grupo Luzada, demostra que hai
un interese real por compartir coñecemento máis aló das aulas. Estes esforzos
son de inestimable valor, pois axudan a achegar a Física a un público máis
amplo e a espertar vocacións entre os futuros estudantes. \textit{Momentum}
pode ser unha plataforma adicional para a divulgación, na que tanto profesorado
como estudantado compartan os seus coñecementos e reflexións sobre os avances
da Física e o seu impacto na sociedade.

Os grupos de investigación da Facultade abranguen moitos eidos do saber actual,
e entre as nosas vetustas paredes poderemos atopar unha opinión formada e
relevante sobre case todas as grandes cuestións científicas actuais nas que a
Física ten algo que dicir. Cuestións como as fontes de enerxía, o cambio
climático, o papel da Ciencia na sociedade tecnolóxica, o crecente liderado da
muller na ciencia e na sociedade, entre outros temas, poden atopar respostas
acudindo ao persoal da Facultade. \textit{Momentum} pode ser un espazo onde
estas cuestións se traten con rigor, pero tamén con proximidade, facendo
accesibles temas complexos para todo o estudantado.

Tamén pode ser un foro para debater sobre que mellorar na Facultade, tanto no
que compete aos estudos en si como á propia organización da Facultade. Unha
revista deste tipo pode servir de punto de encontro para as distintas opinións
e sensibilidades que a conforman: dende o persoal de administración e servizos,
até o estudantado, investigadores e profesores. Crear un espazo onde se poidan
expresar ideas e propostas de mellora pode contribuír a fortalecer a comunidade
universitaria e a facer da Facultade un lugar mellor para todos.

A Facultade ten moita historia que contar. Aquí traballaron investigadores que
foron primeiros autores de artigos que levaron ao Nobel, por aquí pasaron
ducias de Premios Nobel, e ata ¡un cosmonauta ruso! Nas súas aulas forxáronse
moitas carreiras académicas, pero tamén carreiras políticas, de éxito
empresarial millonario e mesmo estrelas da televisión. Recuperar estas
historias, darlles visibilidade e compartilas co estudantado actual pode ser
unha maneira de reforzar o sentimento de pertenza á Facultade e de inspirar ás
novas xeracións.

En definitiva, \textit{Momentum} será aquilo que o estudantado decida que sexa.
A súa riqueza e utilidade dependerán da implicación e da creatividade de quen a
faga posible. Desde o equipo decanal queremos expresar o noso total apoio a
esta iniciativa e animamos a toda a comunidade da Facultade a contribuír a ela,
xa sexa escribindo artigos, propoñendo temas ou simplemente lendo e comentando
os contidos. Estamos seguros de que \textit{Momentum} se converterá nun
elemento fundamental da vida académica da Facultade e desexámosvos moito éxito
nesta aventura. ¡Boa sorte con esta iniciativa!

\end{multicols}
         \newpage
\Titular*%
{Carathéodory e a axiomatización da termodinámica}
{Sebastián Táboas Pazo}
{divulgacion}
{Sobre a pretensión de transformar un coñecemento puramente fenomenolóxico en
puramente matemático.}


\begin{refsection}
\begin{multicols}{2}

\subsection*{Introdución}

Quizais o espírito que une a todos os científicos sexa a pescuda de respostas á
realidade material que se nos impón, na física algúns acadaran o gozo na idea
dunha \textit{teoría do todo} que nos brinde explicación a todo o que acontece.
Outros, no entanto, pretenderan buscar un denominador común capaz de
fundamentar calquera teoría, mesmo \textit{unha teoría do todo}; este é un
espírito axiomatizador que se xesta dende a antigüidade xa con Aristóteles ou
Euclides e se estende até hoxe en día, así tamén se pretendeu axiomatizar a
física. Na conferencia do Congreso internacional de Matemáticos de 1900, David
Hilbert propuxo os coñecidos \textit{23 problemas de
Hilbert}\footnote{Realmente só propuxera dez deles.}, entre eles o de
axiomatizar toda a física.

Os esforzos neste proxecto son moitos e en moitas ramas, mais agora destacamos
o traballo da axiomatización da termodinámica, quizais o menos frutífero de
todos eles. Nun primeiro momento, a persoa que aceptou este desafío foi o
matemático grego Constantin Carathéodory na súa obra \textit{Untersuchungen
über die Grundlagen der Thermodynamik}

\subsection*{A axiomatización de Carathéodory}

No seu traballo de 1909, Carathéodory parte de tan só tres definicións
primitivas e dous axiomas para fundamentar toda a termodinámica que pioneiros
coma Kelvin, Carnot ou Clausius edificaran ao longo de máis de medio século.
Primeiramente, o matemático define a equivalencia de sistemas e as variábeis de
estado, como a caracterización dos cambios de estado. O primeiro axioma que
propón é unha reescritura do primeiro principio da termodinámica:

\begin{quotation}
\textit{Cada fase $\phi_i$ dun sistema S en equilibrio asóciase a unha función
$\varepsilon_i$ das cantidades $V_i$, $p_i$, $m_i$, que é proporcional ao
volume total $V_i$ da fase e denomínase enerxía interna da devandita fase. A
suma $\varepsilon = \sum\varepsilon_i$ de todas as fases chámase enerxía
interna do sistema. En calquera cambio de estado adiabático, o cambio de
enerxía incrementado polo traballo externo, $A$, é nulo; é dicir, en signos, se
se denotan os valores inicial e final da enerxía con} $\varepsilon$ \textit{e}
$\hat{\varepsilon} ~\textit{respectivamente}$:  $$\hat{\varepsilon}
–\varepsilon + A = 0$$ \hspace{1cm}\textit{Carathéodory, 1909}
\end{quotation}

Até aquí non hai nada novidoso, non obstante, debemos pensar que as expresións
matemáticas que frecuentemente manexamos en termodinámica son diferenciais de
Pfaff: $df=\sum x_i dX_i$, os cales no caso adiabático ($\dbar Q = 0$) supoñen
un problema de curvas características entre o estado inicial e o final. Nesta
clase de procesos, estas 1–formas verifican sempre o segundo lema de Schwarz,
polo que se trata de diferenciais exactos que podemos integrar; agora ben, en
procesos non adiabáticos isto non é así. O matemático grego é tamén autor dun
teorema que leva o seu nome: este asegúranos que para toda 1–forma non exacta
existe un factor integrante que a converte en integrábel. Así, afirma, ao
retirar a restricción adiabática, que para $\dbar Q = dU – \dbar W$ existe un
factor integrante, que é o inverso da temperatura absoluta ($1/T$), que
converte a $\dbar Q$ en exacta, este novo diferencial é a entropía $dS$. Desta
forma, semella sólido enunciar o seguinte axioma:

\begin{quotation}
<<En calquera entorno dun estado inicial arbitrario hai estados que son
inaccesíbeis mediante cambios de estado adiabáticos.>> (Carathéodory, 1909).
\end{quotation}

Este teorema dinos que para todo estado $\alpha$ de equilibrio existe polo
menos un estado $\beta$ tal que o segundo é accesíbel dende o primeiro, mais
non á inversa. Isto supón asumir que existe un número arbitrario de curvas
adiabáticas, tantas como se propoña para soster o seu sistema. Até este
momento, Carathéodory non se referira todavía a unha magnitude tan fundamental
na termodinámica como é o calor. Tal é que non fai referencia algunha á noción
primitiva do calor ao longo de todo o seu desenvolvemento, deslindándose da
realidade física evidencíabel e usándoa só como un comodín para describir
aquelas curvas que non foran da clase que define o primeiro postulado.

\subsection*{Críticas}

Estes axiomas semellan constituír un análogo aos principios clásicos da
imposibilidade dos móbiles perpetuos, no caso de seren formulados sen ningunha
clase de erro. A difusión deste traballo foi prácticamente nula nun principio,
non foi até que M. Born sinala o innovador deste artigo que pasara despercibido
nun extenso estudo de 1921. A crítica pronto florece e florea ampla e
negativamente o traballo de Carathéodory, dando un salto exponencial a mediados
do século pasado co auxe da termodinámica irreversíbel na que se evidenciaban
os problemas destes enunciados, cando menos insuficientes, pois o matemático
sempre restrinxírase a un marco de procesos quasiestáticos.

\begin{center}
    \includegraphics[width=0.45\linewidth]{revistas/001/imaxes/termodinamica.jpeg}
    \captionof{figure}{Constantine Carathéodory.}
\end{center}


Unha das primeiras críticas que aparece vén de man de Max Planck, quen sinalara
a súa falta de contacto coa evidencia experimental nunha rama de fundamento
plenamente fenomenolóxica. Este último sentenciara: <<\textit{…ninguén até
agora tratou de alcanzar soamente mediante procesos adiabáticos todo punto no
entorno de calquera estado de equilibrio, e así comprobar se de veras son
inaccesíbeis, […], este axioma non nos brinda o menor indicio que nos permita
diferenciar os estados accesíbeis dos inaccesíbeis.}>> (Planck, 1926). Outra
crítica de entre moitas radica, aínda que se evidencie a existencia dun factor
integrante vinculado á escala absoluta Kelvin, este non comporta necesariamente
unha función da clase das temperaturas termodinámicas, malia ser este erro
rectificábel.

A pesar de nunca ser realmente aceptado (Chandrasekar, no seu libro
\textit{Introducción ao estudo da Estructura Estelar}, denomínao só como ``un
punto de vista''), o simple atreverse é un paso xigante para o desenvolvemento
científico. Este traballo de 1909 constitúe indudabelmente un fito histórico da
ciencia, aínda que sexa errado, relegado á posición de mera curiosidade
académica, acabando por ocupar un lugar secundario nos manuais de
termodinámica.

Ademais desta versión histórica da axiomática termodinámica, atopamos hoxe o
edificante escrito de Herbert B. Callen, \textit{Thermodynamics and an
Introduction to Thermostatistics}, que a estructura axiomáticamente, como outro
punto de vista, e o fundamenta todo no que el chama ``o problema básico da
termodinámica''. Moitos dos textos actuais que empregan estudantes son deudores
de Callen, como por exemplo \textit{Curso sobre el formalismo y los métodos de
la Termodinámica} de J. Biel-Gayé. Aínda con todo, até agora non se conseguiu
unha formulación axiomática puramente matemática equivalente á clásica
formulación de Gibbs exclusivamente fenomenolóxica, quizais, a termodinámica
comporte un caso especial dentro do paradigma científico actual no que é
imposíbel evidenciala dende uns principios matemáticos xerais e primitivos.

\nocite{berenguer.raa_2014}
\nocite{a.lp.berberan-santos.mn_1999}
\nocite{caratheodory.c_1909}
\nocite{planck.m_1926}
\nocite{callen.hb_1960}
\printbibliography

\end{multicols}
\end{refsection}
     \newpage
\Titular%
{Física dun bo almorzo: Chulas perfectas e cafés rebeldes}
{Ánxel Costas Castro}
{divulgacion}
{A física aplicada á cocción e vibracións do café.}


\begin{multicols}{2}

Meus ben queridos lectores, seguro que para moitos dos que estades lendo este
artigo o almorzo sexa unha das comidas máis gozadas do día (ou, polo menos, a
que necesita denantes de ser persoa) e seguramente non queredes un almorzo
ordinario, senón que queredes o mellor posíbel. Permitídeme, pois, axudarvos
nesta tarefa, a priori sinxela, pero que agocha máis física da que pensades. É
lóxico pensar se existe unha fórmula secreta para conseguir noso obxectivo. Por
desgraza, esta fórmula non existe, xa que cada un ten particularidades e gostos
de seu que fan que non exista o almorzo ideal único para todos, pero podemos
axudar a obter, por exemplo, unhas chulas perfectas.

Para iso teremos que saber que factores afectan á cociña deste rico alimento e
así, cada un poida axustar estas variábeis aos seus gostos. Nun acto tan
sinxelo como este, actúan a mecánica de fluídos, a termodinámica e a dinámica.

Para comezar, podemos considerar (creo que nisto estaremos tanto autor como
lector de acordo) que unha chula excelente ten unha forma e grosor uniforme, e
é esponxosa e dourada por fóra ao cocerse homoxeneamente. Para conseguir o
primeiro destes requisitos necesitamos esparcir a masa da chula adecuadamente
na tixola, pero se a masa é moi líquida esparcirase moi rápido e a chula
quedará demasiado delgada e, no caso de que sexa moi espesa, apenas poderá
moverse e quedará irregular. Para conseguir o equilibrio entre estas dúas
situacións teremos que traballar coa viscosidade, que, como todo físico que se
aprecie sabe (seguro que ti tamén o es), é a resistencia dun fluído a fluír. A
proporción dos nosos ingredientes cambiarán esta propiedade, xa que se temos
máis leite ou auga a viscosidade se reducirá, pero non é este o único factor,
pois a presenza de burbullas cando batimos demasiado cambia a fluidez, ou
mesmamente se aumentamos a temperatura, a viscosidade diminuirá.

Unha vez que consigamos a viscosidade desexada virá a cocción da chula, onde a
trasnferencia de calor principalmente será por conducción e convección. Na
conducción, o calor da tixola pasa á masa en contacto e vai determinar como de
rápido se forma a corteza dourada. Por outro lado, a convección será o calor
que vai dende a base da masa até a superficie superior; se non temos unha boa
conducción, poderíamos queimar a base antes que o interior estea ben feito.
Nestes procesos térmicos, a temperatura da tixola é chave. Unha temperatura
óptima sería entre 175 e 200 º$C$, ademais precisamos que a temperatura sexa
uniforme e non existan focos fríos e quentes (científicos con moito tempo libre
viron que se a temperatura da tixola é uniforme a probabilidade de queimar as
chulas caía nun 50 \%). Un bo material pode ser o ferro fundido, pois o teflón
pode quentar de xeito desigual ou unha tixola de aluminio pode perder
temperatura facilmente.

\begin{center}
    \includegraphics*[width=0.7\linewidth]{revistas/001/imaxes/cafeconleche.jpeg}
    \captionof{figure}{\small Mancha dun cafe derramado.}
\end{center}

Agora que xa temos a masa ideal e a temperatura e material da tixola máis
óptimas, chegamos a un punto crítico da nosa receta. Darlle a volta á masa (non
entres en pánico). Se es unha persoa precavida seguramente queiras axudarte de
utensilios de cociña, mais se queres impresionar a alguén, ou se simplemente
queres asemellarte a un gran chef, é probábel que queiras darlle a volta cun
golpe de pulso. Aquí hai que ter en conta a rotación e o momento angular,
ademais de controlar a nosa forza, xa que se nos pasamos, a chula xirará
demasiado rápido e caerá fóra. A aceleración angular máis óptima é de 5
$rad/s^2$, pero como isto é moi difícil de saber a ollo, podo dicirche que a
masa seguirá un movemento parabólico, para que, se te pasas ou quedas curto de
forza, saibas onde vai caer. Unha vez consigas dominar todo este proceso, xa
poderás degustar unhas chulas para lamber os dedos.

Por suposto, este delicioso almorzo non estaría completo sen o líquido
fundamental para o funcionamento correcto dun físico: o café. Mais hai que ter
coidado, porque un paso en falso, literalmente, podería arruinar o noso almorzo
ideal. Seguro que en máis dunha ocasión, mentres levávades o voso café ou volo
traían á vosa mesa da cafetaría, este acababa derramándose polos bordos da
taza e seguramente máis dunha vez pensástedes <<\textit{que patoso son}>> ou
<<\textit{non pode ser tan complicado levar o café sen derramalo}>>. Permíteme
dicirche a ti, meu ben querido lector, que teño unha notica boa e outra mala:
non es unha persoa patosa pero si que é un tema máis complicado do que pensas.

Normalmente o café derrámase aos 4 ou 5 metros (uns 7 ou 10 pasos), aínda que
poden ser máis se estás atento cando o leves. O motivo detrás deste indesexábel
fenómeno ten que ver coas oscilacións e, sendo máis precisos, coas resonancias,
mais acouga, voucho explicar sinxelamente para que comprendas os motivos de tal
desgraza e poidas evitalo, se es capaz diso claro está. Cando comezamos a
camiñar coa taza, a aceleración dos nosos pasos induce unha amplitude inicial
ao vaivén do café, sendo esta maior canto maior sexa a aceleración. Dito
noutras palabras, é oportuno que comeces a andar a miúdo para que partamos con
marxe e poder chegar até nosa mesa con éxito. Neste punto, se prestas atención
ao movemento do café, verás que a amplitude da oscilación comezará a aumentar e
seguramente non saibas por que; trátase dos nosos pasos. Máis ben do ruído que
xeran os nosos pasos, xa que este contén harmónicos de altas frecuencias que
convirten a taza co café nun sistema oscilatorio asimétrico inestábel. Como,
por desgraza, non podemos voar, teremos que coformarnos con camiñar
concentrados e a modiño para minizar o ruído que poidamos xerar ao camiñar e así
aumentar o tempo antes de que suceda a traxedia que tentamos evitar.

Os líquidos (os nosos preciados cafés) dentro de recipentes cilíndricos (súas
pequerrechas tazas) teñen unhas curvas de resonancia relativamente anchas, por
iso se excitan sinxelamente a pesares de estar lonxe da frecuencia de resonancia
(sempre que se supere un certo umbral de ruído). Este tedioso problema levou
aos físicos a deseñar tazas especiais que foran capaces de evitar o
desafortunado derrame, como tazas flexíbeis que amortiguan o vaivén e eliminan
as oscilacións, ou poñendo aneis concéntricos na parede interior da taza para
eliminar o fluxo de masa asociado á frecuencia de resonancia. Todos estes
modelos quedan no esquecemento xa que, ben sexa por simplicidade e costume ou
ben pola preguiza dos nosos, a forma dunha taza común segue sendo a dominante
hoxe en día, a pesar do seu efecto secundario: derramar o noso café. Con todo
isto, non nos queda máis remedio que andar con coidado e evitando facer
aceleracións moi fortes.

Se con isto que acabas de aprender, aínda así tes a mala sorte, a desgraza ou,
se cabe, a torpeza como para verter este líquido vital para nós, non te
enfades, xa que incluso a partir deste molesto accidente podemos aprender algo
novo antes de comezar a traballar. A gota derramada ou, para ser máis precisos,
a mancha que queda ao secarse garda un gran segredo. Se tes a paciencia e o
tempo, verás que a mancha que deixa o noso pequeno desastre comezará a ter unha
forma peculiar, e é que veremos que esta adopta unha forma de circunferencia en
vez dun circulo completo (supoñemos unha mancha circular, xa sabemos como nos
gosta aos físicos aproximar a casos sinxelos).

A que se debe esta rareza? Pois, cando a pinga de café cae enriba da mesa, os
bordes da mancha son máis estreitos ca o centro, polo que se evaporan máis
rápido. Como na evaporación pérdese líquido, aparece un fluxo capilar para
compensar esta perda, o cal arrastra consigo tamén ás partículas de café que
acaban acumulándose nos extremos. Deste xeito temos o patrón de aneis que tanto
nos chamou a atención. Este fenómeno coñécese como \textit{mancha de café} ou
efecto \textit{coffee ring}, e pode darse tamén (ademais de no café) en tinta,
sangue ou líquidos que conteñan partículas en suspensión.

Sabendo isto, se o lector é unha persoa astuta, intuirá que isto pode ter
aplicacións prácticas, e así é. No campo da medicina, sabendo como se evaporan
as gotas de sangue ou outros fluídos biolóxicos, poden crear un diagnóstico; no
campo dos materiais e da nanotecnoloxía, estúdase este fenómeno para mellorar a
distribución de nanopartículas sobre superficies co obxectivo de construír
sensores ou dispositivos electrónicos, como por exemplo paneis solares. Por
suposto, hai moitísimos máis campos e aplicacións para este curioso efecto.
Poucas veces estudar unha mancha resultou tan interesante e tan rentábel!

E, agora si, este querido lector xa poderá tomar o seu almorzo tranquilo, coa
seguridade de que deprendeu a cociñar as chulas ideais, como evitar derramar o
seu importantísimo café matinal e incluso como aproveitar unha mancha no noso
beneficio. Quizais por todo isto, e por máis, dise que o almorzo é a comida
máis importante do día. Que aproveite!

\end{multicols}
          \newpage
\Titular{Skyrmións, ou que é a física}
{Víctor Díaz Díaz}
{divulgacion}
{O que unha teoría errada da interacción forte nos recorda sobre como funciona
a ciencia.}

\begin{multicols}{2}

\subsection*{Introdución}

Hai pouco máis de dous séculos a ciencia comezou a establecerse como unha
ferramenta moi poderosa á hora de investigar e entender o Universo. Grazas a
ela podemos entender unha multitude de fenómenos, e aplicar estes coñecementos
para conseguir fins que doutro xeito parecerían imposibles. Neste ambiente de
éxito, é sinxelo deixarse levar e asumir que a ciencia nos proporciona acceso á
<<\textit{verdade obxectiva}>> do Universo, e que con ela poderemos dar
resposta a calquera pregunta que se nos ocorra. Sen embargo, a realidade é que
a ciencia está limitada por definición.

Neste pequeno artigo imos exemplificar isto no caso da física empregando os
skyrmións.

\subsection*{Os skyrmións}

A física de altas enerxías estuda as compoñentes máis fundamentais da Natureza,
o que na actualidade se corresponde con partículas subatómicas coma os
electróns e os quarks. Dende o punto de vista teórico, estas partículas
represéntanse matematicamente coma solucións dunha determinada teoría cuántica
de campos (QFT), sendo a día de hoxe o Modelo Estándar a teoría máis completa e
efectiva para isto. Estas QFTs son modelos matemáticos moi ricos, que describen
unha gran cantidade de fenómenos a partir dun número relativamente reducido de
ingredientes.

Un exemplo disto pode verse no tipo de solucións destas teorías. Como xa se
dixo, as partículas elementais correspóndense con algunhas das solucións, pero
non con todas. En algúns casos existen outro tipo de solucións, denominadas
\textit{solitóns}, con características similares ás partículas pero tamén con
algunhas diferenzas importantes.

Hoxe en día, os solitóns non gozan do mesmo protagonismo que as partículas á
hora de representar as interaccións fundamentais, pero cando o campo da física
de altas enerxías aínda non estaba tan asentado coma agora, a situación era moi
diferente. Cando aínda se estaba intentando formular teoricamente a interacción
forte, aquela que sucede entre protóns e neutróns (chamados \textit{nucleóns}
en conxunto) no núcleo dos átomos, Tony Skyrme traballou nun modelo que contiña
un campo escalar, o campo piónico, cuxas solucións tipo partícula se
correspondían cos pións e cuxas solucións tipo solitón, chamadas
\textit{skyrmións}, se correspondían cos nucleóns. Así naceu o \textit{modelo
de Skyrme}.

\begin{center}
    \includegraphics*[width=0.7\linewidth]{revistas/001/imaxes/Skyrmions.jpeg}
    \captionof{figure}{Representación gráfica dun skyrmión,onde o número de
buracos se corresponde coa súa carga topolóxica.}
\end{center}

Co tempo, a pesar duns primeiros resultados exitosos da teoría, o modelo de
Skyrme foi abandonado como candidato á teoría fundamental da interacción forte.
Isto débese a que a teoría ignora a existencia dos quarks, as partículas
constituíntes dos nucleóns que foron descubertas en experimentos posteriores e
que son descritas matematicamente pola Cromodinámica Cuántica (QCD) e
subsecuentemente polo Modelo Estándar: a mellor teoría das interaccións
fundamentais (cuánticas) da que dispoñemos a día de hoxe. Esto significa que o
modelo de Skyrme é incompatible co Modelo Estándar, e sen embargo algunha xente
(incluído o autor deste artigo) segue traballando con el. Por que? Que
utilidade pode ter isto?

Para entender isto debemos primeiro fixarnos nun réxime concreto. A baixas
enerxías, a presenza dos quarks apenas ten efecto e os nucleóns compórtanse
como obxectos sen partes, sen graos de liberdade internos. Neste caso, os
cálculos feitos empregando skyrmións proporcionan resultados en bo acordo con
observacións experimentais. Polo tanto, neste réxime é perfectamente válido
empregar o modelo de Skyrme para calcular teoricamente certas magnitudes e
ignorar por completo a QCD. Pero como pode xustificarse esto? Para que queremos
traballar con partículas que non existen podendo facelo cas que si?

\subsection*{Teorías efectivas}

Para responder a esta pregunta, imos considerar
un problema totalmente diferente. Supoñamos que quero describir o movemento de
Xúpiter ao redor do Sol. Necesito coñecer con precisión infinita a dinámica de
todos os obxectos que existen no Universo para poder facelo? Por sorte para
nós, a resposta é non. Non necesitamos coñecer a posición de todas as
partículas de gas que forman o planeta, ou que efecto ten sobre a súa órbita a
variación da súa temperatura debida ao reflexo da luz do Sol sobre os seus
satélites. Basta con considerar os aspectos máis relevantes para a súa
dinámica, que se reducen a pouco máis que a súa masa, radio e distancia ao Sol.
O resto de factores terán o seu efecto, pero será tan pequeno que á hora de
medilo non seremos capaces de resolvelo. Así, podemos ignoralos completamente.

Para facer física, o primeiro paso é elixir que magnitudes son relevantes no
noso problema. Este é un proceso complicado, e sempre en continua revisión,
pero inevitable se queremos poder formular preguntas e buscar respostas que
sexan satisfactorias en certo grao. Con elo, formulamos unha teoría o máis
simple posible que nos resolva as dúbidas que temos, e unha vez que o
conseguimos buscamos predicir con ela fenómenos novos. Pouco a pouco, as
teorías vanse refinando e conséguese explicar unha cantidade cada vez maior de
fenómenos co menor número de ingredientes. Sen embargo, as teorías que quedan
atrás non se esquecen por completo, nin só se formulan teorías novas que
busquen ser máis completas que as anteriores. As teorías que só son válidas nun
rango concreto coñécense como \textit{teorías efectivas} e, nese rango, son
exactamente igual de válidas que as teorías máis xerais.

Se quixese medir a lonxitude da miña mesa cunha precisión de centímetros, sería
o mesmo medir cunha regra con marcas cada milímetro que cunha regra con marcas
cada nanómetro. A resposta sería igualmente 100 centímetros, e calquera decimal
adicional sería descartado, pero é máis sinxelo de ver na regra que está en
milímetros. Esta é precisamente a utilidade de empregar teorías efectivas:
usualmente, cantos menos graos de liberdade teña unha teoría, máis sinxelo será
traballar con ela. E se nos proporciona a precisión que buscamos, o máis lóxico
é traballar ca teoría máis simple. Porén, para describir algunhas propiedades
xerais dun núcleo atómico, podo seguir empregando o modelo de Skyrme porque
este é moito máis sinxelo que a QCD, e os resultados que se obteñen son
suficientemente bos.

\subsection*{A ciencia fala do <<como>>}

A partir do que xa se mencionou debería ser sinxelo clasificar as teorías
efectivas coma <<non reais>>, <<útiles, pero non verdadeiras>>. A pregunta
entón é <<que teorías son reais?>>, ao que podemos responder <<as que non sexan
efectivas>>. Pero resulta que todas as teorías da física son teorías efectivas.
Dende a lei de Hooke ata o Modelo Estándar, só son válidas nun rango
determinado dos seus parámetros. Así, non temos moita xustificación para
asegurar que os quarks son reais e os skyrmións non. Si podemos establecer un
criterio para elixir que teoría é <<máis real>> que outra, considerando que a
<<máis real>> é aquela que describe unha maior cantidade de fenómenos. Sen
embargo, a única <<teoría real>> sería aquela capaz de describir todos os
fenómenos do Universo en conxunto, e isto é algo que non coñecemos e
posiblemente non poidamos formular nunca.

Con isto, vemos de xeito claro unha noción fundamental sobre que é a física (ou
calquera outra ciencia natural) e cal é a súa utilidade. En última instancia, a
física non se preocupa de <<que son as cousas>> se non de <<como se comportan
as cousas>>. Así, dentro dun rango de precisión determinado, un <<núcleo
atómico>> e <<algo que se comporta exactamente igual que un núcleo atómico>>
son nocións totalmente indistinguibles dende o punto de vista científico.
Precisamente neste feito radica a súa eficacia: ao eliminar certos detalles e
centrarse nos aspectos que son fundamentalmente diferentes, resulta moito máis
sinxelo caracterizalos e entendelos.

E é que ao facer física sempre se deben asumir certas cousas, sempre se debe
tomar certos elementos ou relacións como <<reais>>, e a partir delas estudar as
consecuencias. Pero este primeiro paso é un acto non científico. Porén, non
debe menosprezarse a importancia da filosofía no proceso da investigación da
realidade e a súa conexión ca ciencia. Pode parecer que as medidas
experimentais son obxectivas e independentes da opinión de quen as faga, pero
non se debe esquecer que no traballo científico ten unha gran importancia a
interpretación dos resultados e que os principios filosóficos aceptados no
momento guían de xeito sutil a investigación científica. En moitos casos, un
gran avance na ciencia vén despois dun cambio de paradigma filosófico, e
viceversa.

\end{multicols}
        \newpage
\Titular*%
{Coñecendo o Observatorio Ramón María Aller}
{Celia Álvarez Álvarez}
{entrevistas}
{Unha conversa sobre o Observatorio e a Astronomía con J.A. Docobo Durántez.}


\begin{multicols}{2}

\paragraph{P:}

\textit{ Queremos comezar por coñecer un pouco máis de cerca o Observatorio.
Baixo total descoñecemento das rutinas e das labores dos investigadores que
traballan neste observatorio preguntamos: Que actividades se realizan a diario
no observatorio? Con que visión se construiu e con que obxectivos en mente? }

\paragraph{R:}

Son múltiples as actividades. Sempre poño o símil da mesa de
catro patas: investigación, docencia, divulgación astronómica e sección
meteorolóxica. A día de hoxe, e logo de décadas de intenso traballo para
acadalo, o Observatorio é un referente mundial no estudo das estrelas dobres e
múltiples dende diversos puntos de vista, pero temos máis liñas de traballo que
ás veces compartimos con outros colegas. Por exemplo, o estudo da dinámica de
exoplanetas e exosatélites, bólidos, da mecánica celeste, etc. Loitamos arreo
para que o Observatorio sexa a día de hoxe un centro cuns recursos didácticos
moi dignos e na faceta divulgativa foron e son multitude as iniciativas
promovidas, non só para recibir moreas de visitas didácticas escolares senón
tamén para levar, mediante programas financiados por distintos organismos, a
Astronomía a centos de lugares espallados por toda Galicia e incluso fóra dela.
A nosa estación meteorolóxica data de 1947, e na actualidade é a que dispón da
serie de datos máis antiga da cidade de Santiago de Compostela. Dende 1981,
data na que me incorporei ao Observatorio, colaboramos coa Agencia Estatal de
Meteorología (AEMET). (...).

Á pregunta de con qué obxectivos se construiu, teño que dicir que o
Observatorio representa a continuación daquel outro que o padre Aller tiña no
seu Lalín natal, e no que fixo un traballo exemplar durante décadas ata que veu
a Santiago como profesor en 1939 e xa logo en 1943 o seu Observatorio se
trasladou a un novo edificio construído no Campus, na entón chamada Residencia.
(...). Así foi como, coa axuda do recentemente creado CSIC, D. Ramón foi
nomeado Director do novo Observatorio o 27 de setembro de 1943. Os obxectivos
eran que o profesor Aller creara escola na universidade, como así foi, e que o
Observatorio se integrara na vida universitaria, pensando sempre en ir
crecendo, pero despois duns anos co centro a pleno rendemento, o frustrado
relevo xeracional fixo que o Observatorio caese nunha etapa decadente entre
1966 e 1981.

\paragraph{P:}

\textit{ Ademais, eu tiven a oportunidade de poder visitar o seu interior e de
ver de cerca un dos telescopios que garda este edificio: Cales son os
telescopios máis destacables do observatorio? }

\paragraph{R:}

O principal telescopio é un reflector Ritchey-Chrétien de 0.62 metros de
apertura construído nunha factoría de San Petersburgo e instalado aquí en
outubro de 2003, e que me custou media vida acadar. Na cúpula pequena está un
Meade de 0.40 m, máis novo. Logo temos telescopios mais pequenos, de 0.20 m,
que son esenciais para facer observacións con grupos numerosos no xardín do
Observatorio e tamén para levar fóra en saídas de divulgación. (...). Aínda que
con menos prestacións na actualidade, a nosa xoia da coroa é o refractor
Steinheil que pertenceu a Ramón María Aller e quen o doou logo ao Observatorio.
(...). Este instrumento cumpre precisamente agora 100 anos e está en perfecto
estado.

\begin{center}
    \includegraphics[width=0.7\linewidth]{revistas/001/imaxes/astronomia.jpeg}
    \captionof{figure}{{\small Ramón María Aller coa súa discípula Antonia
Ferrín Moreiras, a primeira muller que presentou en España unha Tese de
doutoramento en Astronomía.}}
\end{center}

\paragraph{P:}

\textit{ Sobre o telescopio máis antigo do observatorio: Utilízase a miúdo para
realizar observacións? Para que tipo de observacións serve? }

\paragraph{R:}

Como é lóxico, este refractor do que acabo de falar é un
instrumento venerado pola súa idade. Non se usa de xeito cotiá, senón mais ben
como unha peza histórica a conservar pero onde está ubicado agora na sala
meridiana está apto para realizar as observacións que se lle soliciten. É un
luxo contemplar con el a Lúa, os planetas, estrelas dobres, etc., e os
visitantes así o consideran e  agradecen.

\paragraph{P:}

\textit{ Nun campo tan extenso coma a astronomía é común a colaboración entre
diversos observatorios e centros de investigación situados en diferentes partes
do mundo: Por que é importante o traballo en conxunto de científicos con bases
dispersas ao redor do mundo? Cales son as colaboracións internacionais máis
destacables que tivo o observatorio durante os últimos anos? }

\paragraph{R:}

Hoxe en día a ciencia nos seus distintos ámbitos é difícilmente sustentable sen
colaboracións estratéxicas incluso internacionais, por suposto. Hai moito tempo
que temos relacións moi estreitas con astrónomos e observatorios de moitos
países. Especialmente de Francia, Rusia, USA, Chile, Armenia, Italia,
Inglaterra, etc. Mediante proxectos de investigación témonos desprazado como
usuarios de grandes telescopios a ambos hemisferios para obter datos de
posicións relativas e fotometría diferencial de binarias coa técnica de
interferometría speckle. Na actualidade, de feito, temos a nosa cámara eMCCD
depositada no Byurakan Astrophysical Observatory de Armenia para instalala no
telescopio de 2.6m. Tamén estamos a traballar cos nosos colegas chilenos no
telescopio Soar de 4.2m. As nosas campañas e colaboracións co Special
Astrophysical Observatory de Rusia están aparcadas, pero volverán.

\paragraph{P:}

\textit{ Sobre as aplicacións da astronomía no día a día: O foco de
investigación do OARMA está posto no estudo dos sistemas de estrelas binarias.
A pesares de que moitos estudantes ansiamos como mera finalidade do estudo e da
investigación a obtención do coñecemento no seu estado máis puro, hai tamén
quenes se preguntan: Cales son as motivacións para o estudo das estrelas
dobres? }

\paragraph{R:}

As estrelas dobres constitúen unha rica fonte de información
astronómica. A partir do seu estudo accédese a datos tan importantes en
Astrofísica como as masas estelares, tamaños, distancias, perda de masa,
intercambio de masa entre as compoñentes, binarias con compoñentes variables,
de raios X, con compoñentes evolucionadas, etc. É un mundo con moitísimas
posibilidades. Técnicas como a fotometría usada nas binarias eclipsantes ou a
espectroscopia nas binarias espectroscópicas foron as técnicas maioritariamente
empregadas para poder descubrir os exoplanetas a finais do século pasado.
(...).

\paragraph{P:}

\textit{ Pode a investigación de estrelas dobres contribuír ao progreso de
grandes teorías da Física de actualidade? }

\paragraph{R:}

O avance do perihelio do planeta Mercurio é moi lento, pero no
caso de binarias con compoñentes case en contacto é moito máis rápido e se pode
constatar en moito menos tempo. Esta foi unha das predicións da mecánica
relativista. Se nos fixamos en obxectos compactos como estrelas de neutróns ou
incluso buracos negros, os casos de binariedade foron fundamentais para
detectar as ondas gravitatorias. Seguro que no futuro os sistemas dobres e
múltiples sorpréndenos en xeral con novas aplicacións.

\paragraph{P:}

\textit{ Investigando a maiores sobre a súa traxectoria profesional, gustaríame
coñecer un pouco máis sobre a súa relación coa Unión Astronómica Internacional
(IAU). Vostede foi presidente da Comisión de Estrelas Dobres e Múltiples da
IAU: Cales son os principais obxectivos da Comisión 26 dedicada a sistemas de
estrelas dobres e múltiples? }

\paragraph{R:}

Eu fun elixido por votación de tódolos membros, Vicepresidente da Comisión 26
(Double and Multiple stars) para o período 2006-2009, e como é costume o
Vicepresidente ocupa a Presidencia nos tres anos seguintes. No meu caso entre
2009 e 2012. (...). Os seus obxectivos son potenciar a investigación nesta
importante área da astronomía, promovendo reunións, e influíndo nos comités de
asignación do tempo en grandes telescopios para que os programas de observación
de binarias sexan aceptados. Ben é sabido que na actualidade acádase pouco
tempo dado a longa lista de solicitantes. Incluso hai veces nas que só tes
unhas horas para observar os teus obxectos.

\paragraph{P:}

\textit{ En que sentido facilita a labor investigadora en astronomía a
existencia de Organismos como a IAU? }

\paragraph{R:}

O que facilita a labor investigadora máis que os organismos son os contactos
persoais, que efectivamente adoitan acadarse en congresos ás veces organizados
por eses organismos, neste caso a IAU. Estes contactos son esenciais para a
elaboración de traballos conxuntos, para compartir tempo de observación en
grandes telescopios, e en definitiva ter con quen contactar cando necesitas
algo en relación coa túa investigación. A IAU é o organismo que a nivel mundial
coordina a investigación astronómica en tódolos seus campos. É fundamental a
súa existencia, porque un foro de profesionais é o mellor sitio para debater
cuestións importantes. Por exemplo, lémbrome da Asamblea Xeral da IAU en 2006
en Praga cando se decidiu por votación dos asistentes a nova definición de
planeta que deixou fóra a Plutón, o cal dende entón, xunto con Ceres, Eris,
Makemake, e Haumea forman o grupo de planetas ananos.

\begin{center}
    \includegraphics[width=0.7\linewidth]{revistas/001/imaxes/telescopio.jpg}
    \captionof{figure}{{\small Trátase do refractor Steinheil que Ramón María
Aller mercou en Alemaña e que chegara a Lalín de abril de 1925, polo tanto
agora vai cumprir 100 anos con toda saúde. O instrumento estivo en  Lalín ata
1944, cando veu a Santiago de Compostela para ser instalado na cúpula grande.
Finalmente, en 2003, foi baixado a súa ubicación actual na sala meridiana, onde
segue a estar operativo. }}
\end{center}

\paragraph{P:}

\textit{ Sobre o futuro da investigación en astronomía: Como pensa que
evolucionará o panorama de investigación en astronomía nos vindeiros anos?
Cales son os principais retos para os investigadores e astrónomos do futuro? }

\paragraph{R:}

Coido que os retos non van ser moi diferentes aos que vivimos as
xeracións anteriores. A clave do éxito, como en calquera outra ciencia, é estar
o mellor preparado posible en todos os eidos: manexar idiomas, alto nivel en
informática, bos coñecementos de Física e Matemáticas, estar disposto a
pertencer a grupos competitivos alá onde esteas te necesiten, e sobre todo amar
á Astronomía. As oportunidades acaban aparecendo sempre. Actualmente hai
moitísimos máis medios, grandes telescopios e radiotelescopios, unha
inmensidade de datos obtidos por sondas espaciais, bos investigadores e
titores, pero tamén é certo que existe unha competitividade tremenda.

\paragraph{P:}

\textit{Finalizamos cunha reflexión: Sen falar da astronomía máis técnica ou
computacional e remitíndonos á astronomía observacional máis pura. Pode ser ben
certo que os estudantes botemos en falta en nós mesmos esa curiosidade coa que
comezamos os nosos estudos. E pode tamén que a moitos nos gustaría poder
traballar sobre esa intuición e esa destreza tan práctica da que dispoñían (por
exemplo) os antigos astrónomos, a base de observar pacientemente a natureza que
lles rodeaba. En que aspectos considera que os científicos modernos deberíamos
aprender deles? }

\paragraph{R:}

Hoxe todo vai máis rápido. Non son partidario dos graos, cando eran
licenciaturas o alumnado tiña máis tempo para acudir a actos culturais porque
eran cinco anos en vez dos catro de agora. Ter quitado un ano é un erro porque
non se está en mellor sitio que na universidade para aprender, e o tempo de
estar aquí marca para o futuro. Agora todo o mundo está super ocupado e xa nin
se lle ocorre participar noutras cousas. Cando eu puxen en marcha o PECAS en
1997 participaban 400 persoas, e alumnado dos últimos anos ou novos licenciados
axudábanme nas clases de Astronomía do Cuarto Ciclo. Os científicos modernos
temos que ter máis tempo para pensar nun ambiente sosegado.

\vspace{3em}

Pechamos esta entrevista cunha inspiradora cita do astrónomo Carl Sagan:
<<\textit{We are a way for the Cosmos to know itself}>>
\end{multicols}
 \newpage
\Titular*%
{Física e filosofía: Irmás}
{Mauro Garrido Rodríguez}
{filosofia}
{Sobre a física, a filosofía e a carreira}

\begin{multicols}{2}

Por que a maioría dos que estamos nesta carreira entramos en primeiro lugar?
Acaso non o fixemos para lograr entender os estraños fenómenos da cuántica,
lograr comprender, por exemplo, con que motivo se idearon experimentos mentais
como o do gato de Schrödinger? Ou para saber que era a luz realmente,
comprender as leis de Maxwell con profundidade? Ou foi acaso coa inspiración
case kepleriana de, fascinados pola danza cósmica dos planetas ao redor das
súas órbitas, querer chegar a entender o movemento deses xigantes orbes,
suspendidos na negrura do baleiro? Ou quizais tirando máis cara o clásico, non
sería por lograr, con papel e lapis, entender por que a Terra xira sobre si
mesma, quizais a través dun péndulo en risco de rotura en medio e medio da
facultade?

Fáisenos claro que os motivos máis profundos para estudar esta
ciencia son máis filosóficas ca ``enxeñerís'' ou matemático-formais (o cal non
quita que estes aspectos teñan interese), mais, que é o que promociona a nosa
carreira? Podemos asegurar que realmente coñecemos con profundidade os temas
dos que tratamos, ou é que acaso só nos limitamos a resolver crebacabezas, dun
xeito que podía ser totalmente alleo á natureza da realidade física? Mantemos o
suficiente contacto co que realmente estudamos, é dicir, cos experimentos? Ou é
que cando facemos experimentos limitámonos mais ben a recopilar datos durante
horas primando moito máis o ter unha memoria que presentar máis que adquirir
unha intuición sobre o fenómeno?

Semella que o alumnado, quizais
inconscientemente, busca achegarse á física de xeito máis filosófico, isto é,
en primeiro lugar, afonda con profundidade nos conceptos en si, nos
fundamentos, como por exemplo, se estamos dando a relatividade xeral,
preguntarse, baixo o visto: que é realmente o tempo? Ou se é o caso da mecánica
cuántica: é a realidade independente da obsevación? É a función de ondas algo
\textit{real} ou máis ben un mero instrumento matemático? Estas preguntas van
incluso máis aló: explica a física realmente o universo ou só constrúe modelos
útiles? Cales son os límites da física?

Ningunha destas preguntas é nin tan
sequera minimamente tratada no grao: parece que o ``cala e calcula'', lema
característico de comezos da física de partículas e que pretendía silenciar o
cuestionamento fundacional das súas bases, invadiu a física actual tanto nas
aulas como nos propios investigadores. Non se negará a utilidade deste
pensamento na época: logrou que a cuántica chegase  a explicar o máximo rango
de fenómenos sen ter que gastar tempo nunha reflexión filosófica profunda dos
seus piares; así se obtivo a que é posiblemente a teoría con predicións máis
precisas de tódolos tempos. Mais quizais, á marxe de que preguntarse polos
fundamentos é algo que sempre se debería de facer, fose incluso máis proveitoso
para a física fundamental actual cambiar esta filosofía pragmática dada a era
de estancameno na que vivimos, no sentido de que hai múltiples problemas
abertos (e que levan así bastante tempo) para os cales non tivemos aínda unha
gran revolución ao nivel da relatividade ou do nacemento da mecánica cuántica
que os resolvan, como opinan autores como Lee Smolin ou Sabine Hossenfelder.

\begin{center}
    \includegraphics*[width=0.7\linewidth]{revistas/001/imaxes/BohryEinstein.jpeg}
    \captionof{figure}{diálogo entre Einstein e Bohr sobre a física cuántica,
1925. Imaxe da wikipedia.}
\end{center}

Rompamos co tan repetido lema de ``ninguén entende a mecánica cuántica'' para
evitar cuestionar as súas bases; non nos limitemos a calcular cegamente e
busquemos ante todo coñecer de xeito profundo a realidade, o mundo natural,
físico. Como ben resumiría Einstein, un dos grandes filosofo-físicos, na súa
obra divulgativa \textit{A evolución da física} (1938), a gran motivación da
física é de índole metafísica. <<\textit{A través de todos os esforzos, en cada
unha das dramáticas loitas entre as concepcións vellas e as novas, recoñécese o
eterno anhelo de comprender, a crenza sempre firme na harmonía do mundo, crenza
continuamente reforzada polo encontro de obstáculos sempre crecentes cara a súa
comprensión}>>.

\end{multicols}
            \newpage
\Titular*%
{A morte do xeocentrismo: dos gregos a Kepler}
{Santiago González Gómez}
{historia}
{Unha pequena historia do modelo xeocentrista, dos seus defensores e
detractores, dende os seus inicios gregos ata o seu ocaso na Idade Media.}


\begin{refsection}
\begin{multicols}{2}

\begin{quotation}
Espertei coma dun sono, unha nova luz iluminoume
\textit{Johannes Kepler}
\end{quotation}

Nuns tempos onde o avance da ciencia contrasta co auxe das teorías
conspiranoicas como o terraplanismo, cómpre ás veces lembrar como os seres
humanos foron evolucionando o seu pensamento científico. Unha das preguntas
fundamentais na nosa Historia é que lugar ocupamos no Universo, e nesa
discusión cobra especial importancia unha teoría agora xa descartada, pero que
nos acompañou durante milenios: o xeocentrismo.

\subsection*{Os modelos gregos: Platón, Eudoxo e \\ Ptolomeo}

Os inicios das teorías cosmolóxicas sobre a forma do universo son case sempre
de carácter xeocentrista, como é esperable dende un punto de vista lóxico (é a
forma do Universo máis sinxela posible que concorda con observacións básicas
dende a Terra), pero tamén mitolóxico (a raza humana como centro do mundo). Por
exemplo, os exipcios coidaban que o mundo era plano, con Exipto no seu centro.
En Grecia, existiron nas primeiras idades filosóficas diversas teorías sobre a
forma do Universo. A escola pitagórica foi a primeira en suxerir que a Terra
podería non ser o centro, senón orbitar ao redor dun ``lume central'' (que non
do Sol). Estas teorías buscaban máis unha disposición mítica ou filosófica ca
unha modelización dos movementos dos astros visibles.

Cara ao século IV a.C., a teoría máis aceptada entre os filósofos gregos xa era
que a Terra era unha esfera (non plana!) no centro do Universo. A inmobilidade
da Terra parecía indiscutible, pois nesta época razoaban que, se non se
observaba movemento aparente das estrelas, ou estas estaban moito máis lonxe do
que imaxinaban, ou a Terra estaba queda, sendo esta segunda a hipótese
preferida. Por exemplo, na súa obra \textit{A República}, Platón compara o
universo co fuso dunha roca de fiar. Para el, os astros dispóñense como oito
fusaiolas\footnote{Pezas en forma de disco aburacado que se colocan no extremo
inferior do fuso para enfiar.} (de fóra a dentro: as estrelas fixas, Saturno,
Xúpiter, Marte, o Sol, Venus, Mercurio e a Lúa) que xiran arredor da Terra.
Aristóteles tamén defendeu este modelo, facendo especial fincapé no movemento
dos astros, que para el debía de ser circular uniforme.

\begin{center}
    \includegraphics[width=0.6\linewidth]{./revistas/001/imaxes/epiciclo.png}
    \label{fig:epiciclo}
    \captionof{figure}{{\small Órbita con epiciclos, que Kepler chamaba
``pretzels''.}}
\end{center}

O principal problema do sistema defendido por Platón e Aristóteles é que non
encaixaba coas observacións empíricas: os planetas varían a súa velocidade,
cambian de luminosidade, ás veces producen un movemento retrógrado... Platón
propúxolle a Eudoxo de Cnido salvar a súa teoría, e este respondeu formulando o
modelo de esferas homocéntricas. Nel, cada astro situábase nunha esfera
concéntrica coa Terra, movéndose no seu Ecuador, pero esta esfera movíase á súa
vez unida polos seus polos a unha segunda esfera, que á súa vez movíase unida a
unha terceira ou mesmo a unha cuarta; como nunha esfera armilar. O modelo de
Eudoxo precisaba de 27 esferas en total, e foi quen de escollelas de tal xeito
que se replicaba con bastante precisión o movemento celeste. Malia elegante, a
teoría de Eudoxo non foi moi aceptada, sendo preferido o modelo de ciclos e
epiciclos de Apolonio e Hiparco. Nel, as órbitas (ciclos ou deferentes) en
torno ao Sol tiñan dentro outras órbitas (epiciclos), e eran nestas onde se
movían os astros.


Este modelo foi finalmente refinado por Ptolomeo, quen desprazou á Terra con
respecto ao centro das órbitas, de xeito que estas pasaban a ser excéntricas, e
modificou a uniformidade do movemento aristotélico coa introdución dun punto
(distinto da Terra) chamado ecuante con respecto ao cal si era uniforme a
velocidade angular. Aínda que este modelo, como os anteriores, era puramente
xeométrico e non tentaba explicar o por que desta forma das traxectorias, a súa
capacidade de facer predicións elevouno a modelo xeocéntrico por excelencia.

\begin{center}
    \includegraphics[width=0.6\linewidth]{./revistas/001/imaxes/ecuante.png}
    \captionof{figure}{{\small Á dereita, modelo da ecuante de Ptolomeo. Imaxes
de Wikipedia.}}
\end{center}

O xeocentrismo tiña, porén, os seus detractores. Aristarco de Samos, que foi o
primeiro en probar a esfericidade da Terra coa medición de sombras, tamén
empregou eclipses para realizar unhas precarias estimacións sobre o tamaño do
Sol, da Lúa, e das súas distancias á Terra. Aínda que moi trabucadas, serviron
para que Aristarco se decatase de que o Sol era moito maior do que daquela se
pensaba, levándoo a propoñer un modelo heliocéntrico. Outros filósofos menos
radicais propuxeran sistemas nos que Mercurio e Venus xiraban ao redor do Sol
(pois parecían sempre manterse preto del), malia que este o fixera respecto da
Terra. Estas teorías supoñían un avance enorme ao admitir que non todo revolvía
arredor da Humanidade.

\subsection*{As dúbidas árabes. Copérnico e Kepler}

O modelo ptolemaico seguiu sendo empregado durante a Idade Media por astrónomos
e científicos. En Europa, a Igrexa favoreceu o modelo xeocéntrico por concordar
coa teoloxía desenvolta ata daquela. Cada vez que algunha observación non
parecía encaixar co modelo, este era corrixido engadindo máis epiciclos
(esferas dentro de esferas dentro de esferas...) ata que se acadou un número
ridículo deles.

A astronomía florece no mundo árabe, onde se comeza a dubidar do modelo
ptolemaico. Por exemplo, o gran astrónomo Nasir al-Din al-Tusi atopou
inconsistencias no traballo de Ptolomeo, favorecendo un modelo moi complexo
onde as órbitas, aínda que xeradas por circunferencias, non eran circulares.
Ningún destes modelos prosperou, pois aínda sendo dubidoso o modelo ptolemaico,
as súas predicións non podían ser igualadas.

Na Europa cristiá o traballo astronómico de calidade recupérase a finais da
Idade Media, as novas ideas filosóficas humanistas rexeitan o hieratismo
cristián e abren a porta a un Universo diferente. Nicolás de Nusa razoa que se
todo se move, a Terra tamén o ha de facer, aínda que o movemento que admite é
rotacional, non translacional. Johann Müller Rexiomontano compara a Terra cun
espeto de carne xirando nunha fogata (o Sol). Para el, non é o Sol o que
precisa á Terra, como non é o lume o que precisa o espeto, senón ao revés. Este
é precisamente o pensamento necesario para descartar o xeocentrismo, pero
seguíuse sen superar o modelo ptolemaico; este paso darase en 1543 coa
publicación do \textit{De revolutionibus orbium coelestium} de Copérnico.

O polaco Nicolás Copérnico notou durante os seus estudos en Italia que os
pensadores da época non se poñían de acordo no movemento dos planetas, inferiu
que todos debían de estar pasando algo por alto. Tras investigar as teorías
alternativas propostas dende o tempo dos gregos, convenceuse primeiro de que a
rotación da Terra era moito máis lóxica que un movemento dos astros polo ceo a
velocidades vertixinosas. Logo, aprendendo que algúns filósofos consideraran a
posibilidade de que Mercurio e Venus orbiten ao redor do Sol, deu o paso de
suxerir que o resto de planetas, por tanto, tamén a Terra, facían o mesmo.
Aínda que Copérnico chamou moito a atención entre os seus contemporáneos, non
acabou de establecer a súa teoría, pois o seu sistema, malia explicar os
movementos retrógrados e os cambios de luminosidade dos astros de xeito moi
elegante, non melloraba as predicións do sistema ptolemaico. Isto débese a que
Copérnico continuou empregando órbitas circulares, influenciado polo pensamento
aristotélico. Sería Johannes Kepler quen, partindo das meticulosas observacións
astronómicas do seu mestre Tycho Brahe, chegou á conclusión de que as órbitas
debían de ser elípticas, salvando o sistema de Copérnico. Newton foi quen de
explicar as traxectorias de Kepler coa súa teoría da gravitación, dando por fin
unha explicación ao movemento planetario. Púxose o primeiro cravo do ataúde da
teoría xeocéntrica.

\nocite{dreyer.jle_1906}
\nocite{boyer.cb.merzbach.uc_2011}

\printbibliography
\end{multicols}
\end{refsection}
     \newpage

\newgeometry{
    top    = 5mm,  % marxe superior
    left   = 6mm,  % marxe esquerdo
    right  = 6mm,  % marxe dereito
    bottom = 8mm,  % marxe inferior
    nohead = true, % desactivar o encabezado
    nofoot = true, % desactivar o pe de paxina
}

% FONDO %%%%%%%%%%%%%%%%%%%%%%%%%%%%%%%%%%%%%%%%%%%%%%%%%%%%%%%%%%%%%%%%%%%%%%
\AddToShipoutPictureBG*{
    \put(0,-10){
        \parbox[b][1.2\paperheight]{\paperwidth}{
            \vfill
            \centering
            {\transparent{0.05}\includegraphics{logos/botafumeiro.png}}
            \vfill
        }
    }
}
% No caso de poñelo en todas as páxinas, pode pararse con \ClearShipoutPicture
%
% :FACER: imos cambiar este texto en cada revista?
%
{% os textos de agradecemento e similares
\vspace*{2.4cm}
\begin{center}
    \begin{minipage}{0.68\linewidth}
%    \centering
    {\imprimeDespedida }
\end{minipage} \\[1.5cm]

% Estes textos SEGURO que se cambian en cada revista. Hai que metelo nun macro
\begin{minipage}{0.75\linewidth}
    \centering
    { \imprimeAgradecementos }
    \end{minipage}
\end{center}
}

\thispagestyle{empty}
\vspace*{3em}

\vfill
\vspace{-1cm}
\hrulefill

% :FACER: tal vez facer variables cos contidos das URLs do QR e dos contactos..?

% O do QR non sei por qué pero a veces colocase mal. Mirando con
% lua-visual-debug pareceme que crea un numero incrible de boxes que pode que
% toleen a colocacion doutras cousas. Por agora furrula, polos pelos.
\begin{minipage}[c][5cm]{5cm}
        \begin{center}
            Edicións anteriores: \\[2mm]
        \hypersetup{urlcolor=black}
        \qrset{height = 4cm}%
        % Que quede para a posteridade que rickrolleei a Celia e a Sebas con este QR
        \qrcode{\imprimeEdicionsAnteriores}
    \end{center}
\end{minipage}
%
\hfill % non se deben poñer liñas en branco arredor deste \hfill
%
\begin{minipage}[c][5cm]{4cm}
    \begin{center}
        Participa! (WhatsApp) \\[2mm]
        \hypersetup{urlcolor=black}
        \qrset{height = 4cm}%
        % Que quede para a posteridade que rickrolleei a Celia e a Sebas con este QR
        \qrcode{\imprimeWhatsApp}
    \end{center}
\end{minipage}
%
%
\hfill % deste tampouco
%
% :FACER: pasar .eps -> .pdf leva tempo. Igual deberíamos usar o PDF
% directamente
%
% Sobre o logo en formato .eps. O orixinal ven de:
% https://nubeusc.sharepoint.com/sites/servizos-oficina-web/Documentos%20compartidos/Forms/AllItems.aspx?csf=1&web=1&FolderCTID=0x012000441AD9196B55D84292BD3BC4FC87F798&id=%2Fsites%2Fservizos%2Doficina%2Dweb%2FDocumentos%20compartidos%2FImaxe%20corporativa%2FLogotipo%20da%20USC&viewid=a5e177b5%2D7018%2D46d6%2D9352%2Dfc57158bf6b7
% (espero que a ligazón dure) O arquivo .eps debe pasarse a PDF con 'epstopdf'.
% Esto debería facerse automáticamente, supoñendo que existe o executable.
% Prefiro facelo así porque o que comparten da USC ten ese formato
% Co logo da vicerreitoría
\begin{minipage}[c][4cm]{9cm}
\begin{center}
   Co financiamento de: \\[2mm]
    \begin{picture}(5cm,4cm)%
        \put(-60,30){\hbox{\includegraphics[width=11cm]{logos/vicerreitoria-branco-negro.pdf}}}
    \end{picture}
\end{center}
\end{minipage}
%
% Co logo da USC
%\begin{minipage}[c][4cm]{5cm}
%    \begin{picture}(5cm,4cm)%
%        \put(0,0){\hbox{\includegraphics[width=5cm]{logos/usc-branco-negro.eps}}}
%    \end{picture}
%\end{minipage}
\end{document}
