\Titular*%
{Carathéodory e a axiomatización da termodinámica}
{Sebastián Táboas Pazo}
{divulgacion}
{Sobre a pretensión de transformar un coñecemento puramente fenomenolóxico en
puramente matemático.}


\begin{refsection}
\begin{multicols}{2}

\subsection*{Introdución}

Quizais o espírito que une a todos os científicos sexa a pescuda de respostas á
realidade material que se nos impón, na física algúns acadaran o gozo na idea
dunha \textit{teoría do todo} que nos brinde explicación a todo o que acontece.
Outros, no entanto, pretenderan buscar un denominador común capaz de
fundamentar calquera teoría, mesmo \textit{unha teoría do todo}; este é un
espírito axiomatizador que se xesta dende a antigüidade xa con Aristóteles ou
Euclides e se estende até hoxe en día, así tamén se pretendeu axiomatizar a
física. Na conferencia do Congreso internacional de Matemáticos de 1900, David
Hilbert propuxo os coñecidos \textit{23 problemas de
Hilbert}\footnote{Realmente só propuxera dez deles.}, entre eles o de
axiomatizar toda a física.

Os esforzos neste proxecto son moitos e en moitas ramas, mais agora destacamos
o traballo da axiomatización da termodinámica, quizais o menos frutífero de
todos eles. Nun primeiro momento, a persoa que aceptou este desafío foi o
matemático grego Constantin Carathéodory na súa obra \textit{Untersuchungen
über die Grundlagen der Thermodynamik}

\subsection*{A axiomatización de Carathéodory}

No seu traballo de 1909, Carathéodory parte de tan só tres definicións
primitivas e dous axiomas para fundamentar toda a termodinámica que pioneiros
coma Kelvin, Carnot ou Clausius edificaran ao longo de máis de medio século.
Primeiramente, o matemático define a equivalencia de sistemas e as variábeis de
estado, como a caracterización dos cambios de estado. O primeiro axioma que
propón é unha reescritura do primeiro principio da termodinámica:

\begin{quotation}
\textit{Cada fase $\phi_i$ dun sistema S en equilibrio asóciase a unha función
$\varepsilon_i$ das cantidades $V_i$, $p_i$, $m_i$, que é proporcional ao
volume total $V_i$ da fase e denomínase enerxía interna da devandita fase. A
suma $\varepsilon = \sum\varepsilon_i$ de todas as fases chámase enerxía
interna do sistema. En calquera cambio de estado adiabático, o cambio de
enerxía incrementado polo traballo externo, $A$, é nulo; é dicir, en signos, se
se denotan os valores inicial e final da enerxía con} $\varepsilon$ \textit{e}
$\hat{\varepsilon} ~\textit{respectivamente}$:  $$\hat{\varepsilon}
–\varepsilon + A = 0$$ \hspace{1cm}\textit{Carathéodory, 1909}
\end{quotation}

Até aquí non hai nada novidoso, non obstante, debemos pensar que as expresións
matemáticas que frecuentemente manexamos en termodinámica son diferenciais de
Pfaff: $df=\sum x_i dX_i$, os cales no caso adiabático ($\dbar Q = 0$) supoñen
un problema de curvas características entre o estado inicial e o final. Nesta
clase de procesos, estas 1–formas verifican sempre o segundo lema de Schwarz,
polo que se trata de diferenciais exactos que podemos integrar; agora ben, en
procesos non adiabáticos isto non é así. O matemático grego é tamén autor dun
teorema que leva o seu nome: este asegúranos que para toda 1–forma non exacta
existe un factor integrante que a converte en integrábel. Así, afirma, ao
retirar a restricción adiabática, que para $\dbar Q = dU – \dbar W$ existe un
factor integrante, que é o inverso da temperatura absoluta ($1/T$), que
converte a $\dbar Q$ en exacta, este novo diferencial é a entropía $dS$. Desta
forma, semella sólido enunciar o seguinte axioma:

\begin{quotation}
<<En calquera entorno dun estado inicial arbitrario hai estados que son
inaccesíbeis mediante cambios de estado adiabáticos.>> (Carathéodory, 1909).
\end{quotation}

Este teorema dinos que para todo estado $\alpha$ de equilibrio existe polo
menos un estado $\beta$ tal que o segundo é accesíbel dende o primeiro, mais
non á inversa. Isto supón asumir que existe un número arbitrario de curvas
adiabáticas, tantas como se propoña para soster o seu sistema. Até este
momento, Carathéodory non se referira todavía a unha magnitude tan fundamental
na termodinámica como é o calor. Tal é que non fai referencia algunha á noción
primitiva do calor ao longo de todo o seu desenvolvemento, deslindándose da
realidade física evidencíabel e usándoa só como un comodín para describir
aquelas curvas que non foran da clase que define o primeiro postulado.

\subsection*{Críticas}

Estes axiomas semellan constituír un análogo aos principios clásicos da
imposibilidade dos móbiles perpetuos, no caso de seren formulados sen ningunha
clase de erro. A difusión deste traballo foi prácticamente nula nun principio,
non foi até que M. Born sinala o innovador deste artigo que pasara despercibido
nun extenso estudo de 1921. A crítica pronto florece e florea ampla e
negativamente o traballo de Carathéodory, dando un salto exponencial a mediados
do século pasado co auxe da termodinámica irreversíbel na que se evidenciaban
os problemas destes enunciados, cando menos insuficientes, pois o matemático
sempre restrinxírase a un marco de procesos quasiestáticos.

\begin{center}
    \includegraphics[width=0.45\linewidth]{revistas/001/imaxes/termodinamica.jpeg}
    \captionof{figure}{Constantine Carathéodory.}
\end{center}


Unha das primeiras críticas que aparece vén de man de Max Planck, quen sinalara
a súa falta de contacto coa evidencia experimental nunha rama de fundamento
plenamente fenomenolóxica. Este último sentenciara: <<\textit{…ninguén até
agora tratou de alcanzar soamente mediante procesos adiabáticos todo punto no
entorno de calquera estado de equilibrio, e así comprobar se de veras son
inaccesíbeis, […], este axioma non nos brinda o menor indicio que nos permita
diferenciar os estados accesíbeis dos inaccesíbeis.}>> (Planck, 1926). Outra
crítica de entre moitas radica, aínda que se evidencie a existencia dun factor
integrante vinculado á escala absoluta Kelvin, este non comporta necesariamente
unha función da clase das temperaturas termodinámicas, malia ser este erro
rectificábel.

A pesar de nunca ser realmente aceptado (Chandrasekar, no seu libro
\textit{Introducción ao estudo da Estructura Estelar}, denomínao só como ``un
punto de vista''), o simple atreverse é un paso xigante para o desenvolvemento
científico. Este traballo de 1909 constitúe indudabelmente un fito histórico da
ciencia, aínda que sexa errado, relegado á posición de mera curiosidade
académica, acabando por ocupar un lugar secundario nos manuais de
termodinámica.

Ademais desta versión histórica da axiomática termodinámica, atopamos hoxe o
edificante escrito de Herbert B. Callen, \textit{Thermodynamics and an
Introduction to Thermostatistics}, que a estructura axiomáticamente, como outro
punto de vista, e o fundamenta todo no que el chama ``o problema básico da
termodinámica''. Moitos dos textos actuais que empregan estudantes son deudores
de Callen, como por exemplo \textit{Curso sobre el formalismo y los métodos de
la Termodinámica} de J. Biel-Gayé. Aínda con todo, até agora non se conseguiu
unha formulación axiomática puramente matemática equivalente á clásica
formulación de Gibbs exclusivamente fenomenolóxica, quizais, a termodinámica
comporte un caso especial dentro do paradigma científico actual no que é
imposíbel evidenciala dende uns principios matemáticos xerais e primitivos.

\nocite{berenguer.raa_2014}
\nocite{a.lp.berberan-santos.mn_1999}
\nocite{caratheodory.c_1909}
\nocite{planck.m_1926}
\nocite{callen.hb_1960}
\printbibliography

\end{multicols}
\end{refsection}
