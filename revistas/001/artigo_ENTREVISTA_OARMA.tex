\Titular*%
{Coñecendo o Observatorio Ramón María Aller}
{Celia Álvarez Álvarez}
{entrevistas}
{Unha conversa sobre o Observatorio e a Astronomía con J.A. Docobo Durántez.}


\begin{multicols}{2}

\paragraph{P:}

\textit{ Queremos comezar por coñecer un pouco máis de cerca o Observatorio.
Baixo total descoñecemento das rutinas e das labores dos investigadores que
traballan neste observatorio preguntamos: Que actividades se realizan a diario
no observatorio? Con que visión se construiu e con que obxectivos en mente? }

\paragraph{R:}

Son múltiples as actividades. Sempre poño o símil da mesa de
catro patas: investigación, docencia, divulgación astronómica e sección
meteorolóxica. A día de hoxe, e logo de décadas de intenso traballo para
acadalo, o Observatorio é un referente mundial no estudo das estrelas dobres e
múltiples dende diversos puntos de vista, pero temos máis liñas de traballo que
ás veces compartimos con outros colegas. Por exemplo, o estudo da dinámica de
exoplanetas e exosatélites, bólidos, da mecánica celeste, etc. Loitamos arreo
para que o Observatorio sexa a día de hoxe un centro cuns recursos didácticos
moi dignos e na faceta divulgativa foron e son multitude as iniciativas
promovidas, non só para recibir moreas de visitas didácticas escolares senón
tamén para levar, mediante programas financiados por distintos organismos, a
Astronomía a centos de lugares espallados por toda Galicia e incluso fóra dela.
A nosa estación meteorolóxica data de 1947, e na actualidade é a que dispón da
serie de datos máis antiga da cidade de Santiago de Compostela. Dende 1981,
data na que me incorporei ao Observatorio, colaboramos coa Agencia Estatal de
Meteorología (AEMET). (...).

Á pregunta de con qué obxectivos se construiu, teño que dicir que o
Observatorio representa a continuación daquel outro que o padre Aller tiña no
seu Lalín natal, e no que fixo un traballo exemplar durante décadas ata que veu
a Santiago como profesor en 1939 e xa logo en 1943 o seu Observatorio se
trasladou a un novo edificio construído no Campus, na entón chamada Residencia.
(...). Así foi como, coa axuda do recentemente creado CSIC, D. Ramón foi
nomeado Director do novo Observatorio o 27 de setembro de 1943. Os obxectivos
eran que o profesor Aller creara escola na universidade, como así foi, e que o
Observatorio se integrara na vida universitaria, pensando sempre en ir
crecendo, pero despois duns anos co centro a pleno rendemento, o frustrado
relevo xeracional fixo que o Observatorio caese nunha etapa decadente entre
1966 e 1981.

\paragraph{P:}

\textit{ Ademais, eu tiven a oportunidade de poder visitar o seu interior e de
ver de cerca un dos telescopios que garda este edificio: Cales son os
telescopios máis destacables do observatorio? }

\paragraph{R:}

O principal telescopio é un reflector Ritchey-Chrétien de 0.62 metros de
apertura construído nunha factoría de San Petersburgo e instalado aquí en
outubro de 2003, e que me custou media vida acadar. Na cúpula pequena está un
Meade de 0.40 m, máis novo. Logo temos telescopios mais pequenos, de 0.20 m,
que son esenciais para facer observacións con grupos numerosos no xardín do
Observatorio e tamén para levar fóra en saídas de divulgación. (...). Aínda que
con menos prestacións na actualidade, a nosa xoia da coroa é o refractor
Steinheil que pertenceu a Ramón María Aller e quen o doou logo ao Observatorio.
(...). Este instrumento cumpre precisamente agora 100 anos e está en perfecto
estado.

\begin{center}
    \includegraphics[width=0.7\linewidth]{revistas/001/imaxes/astronomia.jpeg}
    \captionof{figure}{{\small Ramón María Aller coa súa discípula Antonia
Ferrín Moreiras, a primeira muller que presentou en España unha Tese de
doutoramento en Astronomía.}}
\end{center}

\paragraph{P:}

\textit{ Sobre o telescopio máis antigo do observatorio: Utilízase a miúdo para
realizar observacións? Para que tipo de observacións serve? }

\paragraph{R:}

Como é lóxico, este refractor do que acabo de falar é un
instrumento venerado pola súa idade. Non se usa de xeito cotiá, senón mais ben
como unha peza histórica a conservar pero onde está ubicado agora na sala
meridiana está apto para realizar as observacións que se lle soliciten. É un
luxo contemplar con el a Lúa, os planetas, estrelas dobres, etc., e os
visitantes así o consideran e  agradecen.

\paragraph{P:}

\textit{ Nun campo tan extenso coma a astronomía é común a colaboración entre
diversos observatorios e centros de investigación situados en diferentes partes
do mundo: Por que é importante o traballo en conxunto de científicos con bases
dispersas ao redor do mundo? Cales son as colaboracións internacionais máis
destacables que tivo o observatorio durante os últimos anos? }

\paragraph{R:}

Hoxe en día a ciencia nos seus distintos ámbitos é difícilmente sustentable sen
colaboracións estratéxicas incluso internacionais, por suposto. Hai moito tempo
que temos relacións moi estreitas con astrónomos e observatorios de moitos
países. Especialmente de Francia, Rusia, USA, Chile, Armenia, Italia,
Inglaterra, etc. Mediante proxectos de investigación témonos desprazado como
usuarios de grandes telescopios a ambos hemisferios para obter datos de
posicións relativas e fotometría diferencial de binarias coa técnica de
interferometría speckle. Na actualidade, de feito, temos a nosa cámara eMCCD
depositada no Byurakan Astrophysical Observatory de Armenia para instalala no
telescopio de 2.6m. Tamén estamos a traballar cos nosos colegas chilenos no
telescopio Soar de 4.2m. As nosas campañas e colaboracións co Special
Astrophysical Observatory de Rusia están aparcadas, pero volverán.

\paragraph{P:}

\textit{ Sobre as aplicacións da astronomía no día a día: O foco de
investigación do OARMA está posto no estudo dos sistemas de estrelas binarias.
A pesares de que moitos estudantes ansiamos como mera finalidade do estudo e da
investigación a obtención do coñecemento no seu estado máis puro, hai tamén
quenes se preguntan: Cales son as motivacións para o estudo das estrelas
dobres? }

\paragraph{R:}

As estrelas dobres constitúen unha rica fonte de información
astronómica. A partir do seu estudo accédese a datos tan importantes en
Astrofísica como as masas estelares, tamaños, distancias, perda de masa,
intercambio de masa entre as compoñentes, binarias con compoñentes variables,
de raios X, con compoñentes evolucionadas, etc. É un mundo con moitísimas
posibilidades. Técnicas como a fotometría usada nas binarias eclipsantes ou a
espectroscopia nas binarias espectroscópicas foron as técnicas maioritariamente
empregadas para poder descubrir os exoplanetas a finais do século pasado.
(...).

\paragraph{P:}

\textit{ Pode a investigación de estrelas dobres contribuír ao progreso de
grandes teorías da Física de actualidade? }

\paragraph{R:}

O avance do perihelio do planeta Mercurio é moi lento, pero no
caso de binarias con compoñentes case en contacto é moito máis rápido e se pode
constatar en moito menos tempo. Esta foi unha das predicións da mecánica
relativista. Se nos fixamos en obxectos compactos como estrelas de neutróns ou
incluso buracos negros, os casos de binariedade foron fundamentais para
detectar as ondas gravitatorias. Seguro que no futuro os sistemas dobres e
múltiples sorpréndenos en xeral con novas aplicacións.

\paragraph{P:}

\textit{ Investigando a maiores sobre a súa traxectoria profesional, gustaríame
coñecer un pouco máis sobre a súa relación coa Unión Astronómica Internacional
(IAU). Vostede foi presidente da Comisión de Estrelas Dobres e Múltiples da
IAU: Cales son os principais obxectivos da Comisión 26 dedicada a sistemas de
estrelas dobres e múltiples? }

\paragraph{R:}

Eu fun elixido por votación de tódolos membros, Vicepresidente da Comisión 26
(Double and Multiple stars) para o período 2006-2009, e como é costume o
Vicepresidente ocupa a Presidencia nos tres anos seguintes. No meu caso entre
2009 e 2012. (...). Os seus obxectivos son potenciar a investigación nesta
importante área da astronomía, promovendo reunións, e influíndo nos comités de
asignación do tempo en grandes telescopios para que os programas de observación
de binarias sexan aceptados. Ben é sabido que na actualidade acádase pouco
tempo dado a longa lista de solicitantes. Incluso hai veces nas que só tes
unhas horas para observar os teus obxectos.

\paragraph{P:}

\textit{ En que sentido facilita a labor investigadora en astronomía a
existencia de Organismos como a IAU? }

\paragraph{R:}

O que facilita a labor investigadora máis que os organismos son os contactos
persoais, que efectivamente adoitan acadarse en congresos ás veces organizados
por eses organismos, neste caso a IAU. Estes contactos son esenciais para a
elaboración de traballos conxuntos, para compartir tempo de observación en
grandes telescopios, e en definitiva ter con quen contactar cando necesitas
algo en relación coa túa investigación. A IAU é o organismo que a nivel mundial
coordina a investigación astronómica en tódolos seus campos. É fundamental a
súa existencia, porque un foro de profesionais é o mellor sitio para debater
cuestións importantes. Por exemplo, lémbrome da Asamblea Xeral da IAU en 2006
en Praga cando se decidiu por votación dos asistentes a nova definición de
planeta que deixou fóra a Plutón, o cal dende entón, xunto con Ceres, Eris,
Makemake, e Haumea forman o grupo de planetas ananos.

\begin{center}
    \includegraphics[width=0.7\linewidth]{revistas/001/imaxes/telescopio.jpg}
    \captionof{figure}{{\small Trátase do refractor Steinheil que Ramón María
Aller mercou en Alemaña e que chegara a Lalín de abril de 1925, polo tanto
agora vai cumprir 100 anos con toda saúde. O instrumento estivo en  Lalín ata
1944, cando veu a Santiago de Compostela para ser instalado na cúpula grande.
Finalmente, en 2003, foi baixado a súa ubicación actual na sala meridiana, onde
segue a estar operativo. }}
\end{center}

\paragraph{P:}

\textit{ Sobre o futuro da investigación en astronomía: Como pensa que
evolucionará o panorama de investigación en astronomía nos vindeiros anos?
Cales son os principais retos para os investigadores e astrónomos do futuro? }

\paragraph{R:}

Coido que os retos non van ser moi diferentes aos que vivimos as
xeracións anteriores. A clave do éxito, como en calquera outra ciencia, é estar
o mellor preparado posible en todos os eidos: manexar idiomas, alto nivel en
informática, bos coñecementos de Física e Matemáticas, estar disposto a
pertencer a grupos competitivos alá onde esteas te necesiten, e sobre todo amar
á Astronomía. As oportunidades acaban aparecendo sempre. Actualmente hai
moitísimos máis medios, grandes telescopios e radiotelescopios, unha
inmensidade de datos obtidos por sondas espaciais, bos investigadores e
titores, pero tamén é certo que existe unha competitividade tremenda.

\paragraph{P:}

\textit{Finalizamos cunha reflexión: Sen falar da astronomía máis técnica ou
computacional e remitíndonos á astronomía observacional máis pura. Pode ser ben
certo que os estudantes botemos en falta en nós mesmos esa curiosidade coa que
comezamos os nosos estudos. E pode tamén que a moitos nos gustaría poder
traballar sobre esa intuición e esa destreza tan práctica da que dispoñían (por
exemplo) os antigos astrónomos, a base de observar pacientemente a natureza que
lles rodeaba. En que aspectos considera que os científicos modernos deberíamos
aprender deles? }

\paragraph{R:}

Hoxe todo vai máis rápido. Non son partidario dos graos, cando eran
licenciaturas o alumnado tiña máis tempo para acudir a actos culturais porque
eran cinco anos en vez dos catro de agora. Ter quitado un ano é un erro porque
non se está en mellor sitio que na universidade para aprender, e o tempo de
estar aquí marca para o futuro. Agora todo o mundo está super ocupado e xa nin
se lle ocorre participar noutras cousas. Cando eu puxen en marcha o PECAS en
1997 participaban 400 persoas, e alumnado dos últimos anos ou novos licenciados
axudábanme nas clases de Astronomía do Cuarto Ciclo. Os científicos modernos
temos que ter máis tempo para pensar nun ambiente sosegado.

\vspace{3em}

Pechamos esta entrevista cunha inspiradora cita do astrónomo Carl Sagan:
<<\textit{We are a way for the Cosmos to know itself}>>
\end{multicols}
