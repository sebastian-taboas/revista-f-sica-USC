\Titular*%
{A morte do xeocentrismo: dos gregos a Kepler}
{Santiago González Gómez}
{historia}
{Unha pequena historia do modelo xeocentrista, dos seus defensores e
detractores, dende os seus inicios gregos ata o seu ocaso na Idade Media.}


\begin{refsection}
\begin{multicols}{2}

\begin{quotation}
Espertei coma dun sono, unha nova luz iluminoume
\textit{Johannes Kepler}
\end{quotation}

Nuns tempos onde o avance da ciencia contrasta co auxe das teorías
conspiranoicas como o terraplanismo, cómpre ás veces lembrar como os seres
humanos foron evolucionando o seu pensamento científico. Unha das preguntas
fundamentais na nosa Historia é que lugar ocupamos no Universo, e nesa
discusión cobra especial importancia unha teoría agora xa descartada, pero que
nos acompañou durante milenios: o xeocentrismo.

\subsection*{Os modelos gregos: Platón, Eudoxo e \\ Ptolomeo}

Os inicios das teorías cosmolóxicas sobre a forma do universo son case sempre
de carácter xeocentrista, como é esperable dende un punto de vista lóxico (é a
forma do Universo máis sinxela posible que concorda con observacións básicas
dende a Terra), pero tamén mitolóxico (a raza humana como centro do mundo). Por
exemplo, os exipcios coidaban que o mundo era plano, con Exipto no seu centro.
En Grecia, existiron nas primeiras idades filosóficas diversas teorías sobre a
forma do Universo. A escola pitagórica foi a primeira en suxerir que a Terra
podería non ser o centro, senón orbitar ao redor dun ``lume central'' (que non
do Sol). Estas teorías buscaban máis unha disposición mítica ou filosófica ca
unha modelización dos movementos dos astros visibles.

Cara ao século IV a.C., a teoría máis aceptada entre os filósofos gregos xa era
que a Terra era unha esfera (non plana!) no centro do Universo. A inmobilidade
da Terra parecía indiscutible, pois nesta época razoaban que, se non se
observaba movemento aparente das estrelas, ou estas estaban moito máis lonxe do
que imaxinaban, ou a Terra estaba queda, sendo esta segunda a hipótese
preferida. Por exemplo, na súa obra \textit{A República}, Platón compara o
universo co fuso dunha roca de fiar. Para el, os astros dispóñense como oito
fusaiolas\footnote{Pezas en forma de disco aburacado que se colocan no extremo
inferior do fuso para enfiar.} (de fóra a dentro: as estrelas fixas, Saturno,
Xúpiter, Marte, o Sol, Venus, Mercurio e a Lúa) que xiran arredor da Terra.
Aristóteles tamén defendeu este modelo, facendo especial fincapé no movemento
dos astros, que para el debía de ser circular uniforme.

\begin{center}
    \includegraphics[width=0.6\linewidth]{./revistas/001/imaxes/epiciclo.png}
    \label{fig:epiciclo}
    \captionof{figure}{{\small Órbita con epiciclos, que Kepler chamaba
``pretzels''.}}
\end{center}

O principal problema do sistema defendido por Platón e Aristóteles é que non
encaixaba coas observacións empíricas: os planetas varían a súa velocidade,
cambian de luminosidade, ás veces producen un movemento retrógrado... Platón
propúxolle a Eudoxo de Cnido salvar a súa teoría, e este respondeu formulando o
modelo de esferas homocéntricas. Nel, cada astro situábase nunha esfera
concéntrica coa Terra, movéndose no seu Ecuador, pero esta esfera movíase á súa
vez unida polos seus polos a unha segunda esfera, que á súa vez movíase unida a
unha terceira ou mesmo a unha cuarta; como nunha esfera armilar. O modelo de
Eudoxo precisaba de 27 esferas en total, e foi quen de escollelas de tal xeito
que se replicaba con bastante precisión o movemento celeste. Malia elegante, a
teoría de Eudoxo non foi moi aceptada, sendo preferido o modelo de ciclos e
epiciclos de Apolonio e Hiparco. Nel, as órbitas (ciclos ou deferentes) en
torno ao Sol tiñan dentro outras órbitas (epiciclos), e eran nestas onde se
movían os astros.


Este modelo foi finalmente refinado por Ptolomeo, quen desprazou á Terra con
respecto ao centro das órbitas, de xeito que estas pasaban a ser excéntricas, e
modificou a uniformidade do movemento aristotélico coa introdución dun punto
(distinto da Terra) chamado ecuante con respecto ao cal si era uniforme a
velocidade angular. Aínda que este modelo, como os anteriores, era puramente
xeométrico e non tentaba explicar o por que desta forma das traxectorias, a súa
capacidade de facer predicións elevouno a modelo xeocéntrico por excelencia.

\begin{center}
    \includegraphics[width=0.6\linewidth]{./revistas/001/imaxes/ecuante.png}
    \captionof{figure}{{\small Á dereita, modelo da ecuante de Ptolomeo. Imaxes
de Wikipedia.}}
\end{center}

O xeocentrismo tiña, porén, os seus detractores. Aristarco de Samos, que foi o
primeiro en probar a esfericidade da Terra coa medición de sombras, tamén
empregou eclipses para realizar unhas precarias estimacións sobre o tamaño do
Sol, da Lúa, e das súas distancias á Terra. Aínda que moi trabucadas, serviron
para que Aristarco se decatase de que o Sol era moito maior do que daquela se
pensaba, levándoo a propoñer un modelo heliocéntrico. Outros filósofos menos
radicais propuxeran sistemas nos que Mercurio e Venus xiraban ao redor do Sol
(pois parecían sempre manterse preto del), malia que este o fixera respecto da
Terra. Estas teorías supoñían un avance enorme ao admitir que non todo revolvía
arredor da Humanidade.

\subsection*{As dúbidas árabes. Copérnico e Kepler}

O modelo ptolemaico seguiu sendo empregado durante a Idade Media por astrónomos
e científicos. En Europa, a Igrexa favoreceu o modelo xeocéntrico por concordar
coa teoloxía desenvolta ata daquela. Cada vez que algunha observación non
parecía encaixar co modelo, este era corrixido engadindo máis epiciclos
(esferas dentro de esferas dentro de esferas...) ata que se acadou un número
ridículo deles.

A astronomía florece no mundo árabe, onde se comeza a dubidar do modelo
ptolemaico. Por exemplo, o gran astrónomo Nasir al-Din al-Tusi atopou
inconsistencias no traballo de Ptolomeo, favorecendo un modelo moi complexo
onde as órbitas, aínda que xeradas por circunferencias, non eran circulares.
Ningún destes modelos prosperou, pois aínda sendo dubidoso o modelo ptolemaico,
as súas predicións non podían ser igualadas.

Na Europa cristiá o traballo astronómico de calidade recupérase a finais da
Idade Media, as novas ideas filosóficas humanistas rexeitan o hieratismo
cristián e abren a porta a un Universo diferente. Nicolás de Nusa razoa que se
todo se move, a Terra tamén o ha de facer, aínda que o movemento que admite é
rotacional, non translacional. Johann Müller Rexiomontano compara a Terra cun
espeto de carne xirando nunha fogata (o Sol). Para el, non é o Sol o que
precisa á Terra, como non é o lume o que precisa o espeto, senón ao revés. Este
é precisamente o pensamento necesario para descartar o xeocentrismo, pero
seguíuse sen superar o modelo ptolemaico; este paso darase en 1543 coa
publicación do \textit{De revolutionibus orbium coelestium} de Copérnico.

O polaco Nicolás Copérnico notou durante os seus estudos en Italia que os
pensadores da época non se poñían de acordo no movemento dos planetas, inferiu
que todos debían de estar pasando algo por alto. Tras investigar as teorías
alternativas propostas dende o tempo dos gregos, convenceuse primeiro de que a
rotación da Terra era moito máis lóxica que un movemento dos astros polo ceo a
velocidades vertixinosas. Logo, aprendendo que algúns filósofos consideraran a
posibilidade de que Mercurio e Venus orbiten ao redor do Sol, deu o paso de
suxerir que o resto de planetas, por tanto, tamén a Terra, facían o mesmo.
Aínda que Copérnico chamou moito a atención entre os seus contemporáneos, non
acabou de establecer a súa teoría, pois o seu sistema, malia explicar os
movementos retrógrados e os cambios de luminosidade dos astros de xeito moi
elegante, non melloraba as predicións do sistema ptolemaico. Isto débese a que
Copérnico continuou empregando órbitas circulares, influenciado polo pensamento
aristotélico. Sería Johannes Kepler quen, partindo das meticulosas observacións
astronómicas do seu mestre Tycho Brahe, chegou á conclusión de que as órbitas
debían de ser elípticas, salvando o sistema de Copérnico. Newton foi quen de
explicar as traxectorias de Kepler coa súa teoría da gravitación, dando por fin
unha explicación ao movemento planetario. Púxose o primeiro cravo do ataúde da
teoría xeocéntrica.

\nocite{dreyer.jle_1906}
\nocite{boyer.cb.merzbach.uc_2011}

\printbibliography
\end{multicols}
\end{refsection}
