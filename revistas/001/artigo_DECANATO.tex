\Titular*%
{Benvida a \textit{MOMENTUM}}%
{Equipo Decanal}%
{profesorado}%
{}%

\vspace{-1.1cm}

\begin{multicols}{2}

Dende o Decanato da Facultade saudamos con entusiasmo a iniciativa de abrir un
novo espazo onde o estudantado poida plasmar as súas inquedanzas e intereses ao
redor da súa estadía na nosa Facultade de Física da USC. A revista
\textit{Momentum} recolle o testemuño de anteriores iniciativas similares que
existiron na Facultade, nomeadamente as revistas A \textit{Gota de Millikan} e
o \textit{Físical Interviú} que gozaron de moito éxito entre os colectivos da
Facultade. Comparten compromiso coa divulgación, xestión e participación
estudiantil e diversidade de contidos. As tres publicacións, ao igual que o QMF
ou a Quantum Cup reactivada este ano, constitúen unha ponte que conecta varias
xeracións de físicos e físicas separadas por case corenta anos. Da súa lectura
se percebe unha esencia común que perdura adaptada á evolución do contexto
socio-económico e científico-tecnolóxico.

A \textit{Gota de Millikan} naceu en 1988, pouco despois que a Facultade, cunha
orientación fundamentalmente divulgadora promovida pola Asociación Isaac
Newton. Publicaban traballos moi rigorosos, asinados tanto polo estudantado
como polo profesorado, que trataban diversas temáticas relacionadas coas
ciencias experimentais e matemáticas. Entre o estudantado asinante podedes
atopar profesorado de sobra coñecido por vós, algún xa xubilado. A \textit{Gota
de Millikan} adicaba unha sección ao xadrez, unha das actividades favoritas do
estudantado de Física da época. Eran míticas as análises das xogadas de Carril
que poñía en práctica na cafetaría de Pachín situada no espazo que ocupa
actualmente a Sala de Lectura. Esta tradición mantívose e consolidouse co tempo
coa celebración periódica de torneos de xadrez na Facultade de Física da man de
Suso Mosqueira que atraen a toda a comunidade universitaria. Incorporaba tamén
unha sección de Paradoxas que invitaba aos lectores a pensar. Para sufragar o
custo que supoñía ser editada en papel, incorporaba anuncios publicitarios de
establecementos vinculados á actividade diurna e nocturna do estudantado que
hoxe espertan un sentimento doce de nostalxia. Publicáronse 9 números entre
1988 e 1992, e un décimo e derradeiro en 1997 que podedes atopar na Biblioteca.
Existiu nunha época na que o acceso a un ordenador persoal non era tan común e
moito menos o acceso ao correo electrónico; nunha época na que as redes sociais
estaban constituídas por esa lista de coñecidos cos que interactuabas na
universidade, no transporte, á hora de xantar en comedores ou bares, nos
locutorios, no tempo de ocio...; rede que se ía ampliando con cada apretón de
mans ou par de bicos tras unha presentación e que se ía consolidando co
contacto periódico pois nin o teléfono fixo estaba ao alcance de todos.

\textit{Físical Interviú} aproximábase máis a un boletín. O seu formato era
máis flexible (entrevistas, opinións, divulgación científica, novas, cartas,
crónica social, crónica política...) e os seus contidos, máis variados, foron
evolucionando co tempo. Os avances nas comunicacións facilitaron o acceso á
información e a persoal científico ou emprendedor alleo á USC que contribuiu
con colaboracións puntuais. En \textit{Físical Interviú} tratábanse temas
relacionados coa vida na Facultade: o edificio, a docencia, a investigación ou
os logros no deporte por parte do estudantado, por exemplo. Constitúe un
verdadeiro documento gráfico que lembra acontecementos e momentos  importantes
como os encontros do alumnado cos Premios Nobel, grazas ao programa ConCiencia;
as distintas edicións do QMF; ou o renacer da Asociación Isaac Newton; a
constitución da Delegación do alumnado, a adaptación ao Espazo Europeo de
Educación Superior. Foi un medio pioneiro na visibilización da muller na Física
co artigo \textit{Mi mamá es investigadora} no ano 2008 e a sección ConCiencia
de Muller nos últimos tempos.. Estivo activa de xeito intermitente ata 2021, un
período moi extenso suxeito a moitos cambios. Os avances en tecnoloxía dixital
e informática permitiron co tempo publicar edicións dixitais, que estaban
dispoñibles nun espazo web, ademais das tiradas impresas. Algúns dos exemplares
impresos  publicados tamén están dispoñibles na Biblioteca para consulta.

O contexto no que xorde \textit{Momentum} é distinto ao doutras épocas, pois as
formas de comunicación evolucionaron de maneira significativa. Hoxe, o espazo
público do debate e da reflexión está amplamente dominado polas redes sociais,
que permiten a interacción inmediata e a difusión rápida de contidos. Porén,
estas plataformas teñen tamén as súas limitacións: a inmediatez moitas veces
non deixa espazo para a reflexión pausada, para a análise en profundidade ou
para o intercambio estruturado de ideas. \textit{Momentum}, pola súa natureza
como publicación periódica, pode suplir esa carencia ao ofrecer un lugar onde
as ideas se expresen con maior desenvolvemento, onde se poidan articular
argumentos sólidos e onde se fomente un diálogo enriquecedor sobre os temas que
máis preocupan e interesan ao estudantado e ao profesorado.

A nosa experiencia dinos que o éxito de \textit{Momentum} será directamente
proporcional á implicación que teña o estudantado á hora de participar nela.
Unha publicación deste tipo pode converterse nun valioso engadido ao conxunto
de espazos de expresión dentro da Facultade. Non só permitirá compartir
coñecementos e experiencias, senón que tamén servirá como rexistro histórico do
que acontece en cada etapa. En anos futuros, as novas xeracións de estudantes
poderán botar unha ollada ás páxinas de \textit{Momentum} e descubrir como eran
os debates, as preocupacións e as ilusións da comunidade universitaria do seu
tempo.

Seguro que abundarán as referencias ás distintas actividades que teñen lugar na
Facultade. Temos a sorte de contar cun estudantado moi activo, que enche o
curso académico con numerosas iniciativas: ceas, concertos como o sacrosanto
Quantum Music Festival (QMF), actos de graduación, actividades culturais e
deportivas, entre outras. Todas estas accións contribúen a crear un ambiente
universitario vivo e dinámico, no que non todo se reduce ás aulas e aos
laboratorios. \textit{Momentum} pode ser o espazo idóneo para recoller estas
experiencias, para narrar as anécdotas que xorden delas e para reflectir a
diversidade de intereses do noso estudantado.

Pero non só sabemos de festa. Desde a Facultade tamén se promoven moitas outras
iniciativas de carácter académico e científico, sendo as máis relevantes os
diversos congresos, cursos e seminarios que teñen lugar ao longo do ano.
Ademais, levamos anos organizando a fase galega da Olimpiada de Física, en
coordinación coa Real Sociedade Española de Física, ofrecendo aos estudantes de
secundaria a oportunidade de achegarse ao mundo da Física con retos e desafíos
intelectuais estimulantes. Unha publicación como \textit{Momentum} pode servir
para documentar e destacar estas actividades, dándolles visibilidade e
recoñecendo o esforzo de quen as fai posibles.

Moitas veces, as cousas que suceden na Facultade pasan desapercibidas porque
estamos inmersos na rutina diaria e non sempre somos conscientes da riqueza da
nosa contorna. \textit{Momentum} pode axudar a cambiar iso, ofrecendo un
escaparate para dar a coñecer proxectos, experiencias e historias que, doutro
xeito, poderían quedar relegadas ao esquecemento. Unha revista feita polo
estudantado e para o estudantado ten o potencial de reflectir a verdadeira
esencia da Facultade de Física: a súa diversidade, a súa vitalidade e o seu
espírito crítico e innovador.

Ademais, \textit{Momentum} pode servir como vehículo para a divulgación
científica dentro da Facultade. Contamos con expertos divulgadores recoñecidos
a nivel nacional e internacional polo seu xeito sinxelo e efectivo de explicar
a Ciencia á poboación xeral. O feito de que o noso estudantado tamén estea
implicado en iniciativas de divulgación, como o grupo Luzada, demostra que hai
un interese real por compartir coñecemento máis aló das aulas. Estes esforzos
son de inestimable valor, pois axudan a achegar a Física a un público máis
amplo e a espertar vocacións entre os futuros estudantes. \textit{Momentum}
pode ser unha plataforma adicional para a divulgación, na que tanto profesorado
como estudantado compartan os seus coñecementos e reflexións sobre os avances
da Física e o seu impacto na sociedade.

Os grupos de investigación da Facultade abranguen moitos eidos do saber actual,
e entre as nosas vetustas paredes poderemos atopar unha opinión formada e
relevante sobre case todas as grandes cuestións científicas actuais nas que a
Física ten algo que dicir. Cuestións como as fontes de enerxía, o cambio
climático, o papel da Ciencia na sociedade tecnolóxica, o crecente liderado da
muller na ciencia e na sociedade, entre outros temas, poden atopar respostas
acudindo ao persoal da Facultade. \textit{Momentum} pode ser un espazo onde
estas cuestións se traten con rigor, pero tamén con proximidade, facendo
accesibles temas complexos para todo o estudantado.

Tamén pode ser un foro para debater sobre que mellorar na Facultade, tanto no
que compete aos estudos en si como á propia organización da Facultade. Unha
revista deste tipo pode servir de punto de encontro para as distintas opinións
e sensibilidades que a conforman: dende o persoal de administración e servizos,
até o estudantado, investigadores e profesores. Crear un espazo onde se poidan
expresar ideas e propostas de mellora pode contribuír a fortalecer a comunidade
universitaria e a facer da Facultade un lugar mellor para todos.

A Facultade ten moita historia que contar. Aquí traballaron investigadores que
foron primeiros autores de artigos que levaron ao Nobel, por aquí pasaron
ducias de Premios Nobel, e ata ¡un cosmonauta ruso! Nas súas aulas forxáronse
moitas carreiras académicas, pero tamén carreiras políticas, de éxito
empresarial millonario e mesmo estrelas da televisión. Recuperar estas
historias, darlles visibilidade e compartilas co estudantado actual pode ser
unha maneira de reforzar o sentimento de pertenza á Facultade e de inspirar ás
novas xeracións.

En definitiva, \textit{Momentum} será aquilo que o estudantado decida que sexa.
A súa riqueza e utilidade dependerán da implicación e da creatividade de quen a
faga posible. Desde o equipo decanal queremos expresar o noso total apoio a
esta iniciativa e animamos a toda a comunidade da Facultade a contribuír a ela,
xa sexa escribindo artigos, propoñendo temas ou simplemente lendo e comentando
os contidos. Estamos seguros de que \textit{Momentum} se converterá nun
elemento fundamental da vida académica da Facultade e desexámosvos moito éxito
nesta aventura. ¡Boa sorte con esta iniciativa!

\end{multicols}
