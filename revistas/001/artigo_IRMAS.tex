\Titular*%
{Física e filosofía: Irmás}
{Mauro Garrido Rodríguez}
{filosofia}
{Sobre a física, a filosofía e a carreira}

\begin{multicols}{2}

Por que a maioría dos que estamos nesta carreira entramos en primeiro lugar?
Acaso non o fixemos para lograr entender os estraños fenómenos da cuántica,
lograr comprender, por exemplo, con que motivo se idearon experimentos mentais
como o do gato de Schrödinger? Ou para saber que era a luz realmente,
comprender as leis de Maxwell con profundidade? Ou foi acaso coa inspiración
case kepleriana de, fascinados pola danza cósmica dos planetas ao redor das
súas órbitas, querer chegar a entender o movemento deses xigantes orbes,
suspendidos na negrura do baleiro? Ou quizais tirando máis cara o clásico, non
sería por lograr, con papel e lapis, entender por que a Terra xira sobre si
mesma, quizais a través dun péndulo en risco de rotura en medio e medio da
facultade?

Fáisenos claro que os motivos máis profundos para estudar esta
ciencia son máis filosóficas ca ``enxeñerís'' ou matemático-formais (o cal non
quita que estes aspectos teñan interese), mais, que é o que promociona a nosa
carreira? Podemos asegurar que realmente coñecemos con profundidade os temas
dos que tratamos, ou é que acaso só nos limitamos a resolver crebacabezas, dun
xeito que podía ser totalmente alleo á natureza da realidade física? Mantemos o
suficiente contacto co que realmente estudamos, é dicir, cos experimentos? Ou é
que cando facemos experimentos limitámonos mais ben a recopilar datos durante
horas primando moito máis o ter unha memoria que presentar máis que adquirir
unha intuición sobre o fenómeno?

Semella que o alumnado, quizais
inconscientemente, busca achegarse á física de xeito máis filosófico, isto é,
en primeiro lugar, afonda con profundidade nos conceptos en si, nos
fundamentos, como por exemplo, se estamos dando a relatividade xeral,
preguntarse, baixo o visto: que é realmente o tempo? Ou se é o caso da mecánica
cuántica: é a realidade independente da obsevación? É a función de ondas algo
\textit{real} ou máis ben un mero instrumento matemático? Estas preguntas van
incluso máis aló: explica a física realmente o universo ou só constrúe modelos
útiles? Cales son os límites da física?

Ningunha destas preguntas é nin tan
sequera minimamente tratada no grao: parece que o ``cala e calcula'', lema
característico de comezos da física de partículas e que pretendía silenciar o
cuestionamento fundacional das súas bases, invadiu a física actual tanto nas
aulas como nos propios investigadores. Non se negará a utilidade deste
pensamento na época: logrou que a cuántica chegase  a explicar o máximo rango
de fenómenos sen ter que gastar tempo nunha reflexión filosófica profunda dos
seus piares; así se obtivo a que é posiblemente a teoría con predicións máis
precisas de tódolos tempos. Mais quizais, á marxe de que preguntarse polos
fundamentos é algo que sempre se debería de facer, fose incluso máis proveitoso
para a física fundamental actual cambiar esta filosofía pragmática dada a era
de estancameno na que vivimos, no sentido de que hai múltiples problemas
abertos (e que levan así bastante tempo) para os cales non tivemos aínda unha
gran revolución ao nivel da relatividade ou do nacemento da mecánica cuántica
que os resolvan, como opinan autores como Lee Smolin ou Sabine Hossenfelder.

\begin{center}
    \includegraphics*[width=0.7\linewidth]{revistas/001/imaxes/BohryEinstein.jpeg}
    \captionof{figure}{diálogo entre Einstein e Bohr sobre a física cuántica,
1925. Imaxe da wikipedia.}
\end{center}

Rompamos co tan repetido lema de ``ninguén entende a mecánica cuántica'' para
evitar cuestionar as súas bases; non nos limitemos a calcular cegamente e
busquemos ante todo coñecer de xeito profundo a realidade, o mundo natural,
físico. Como ben resumiría Einstein, un dos grandes filosofo-físicos, na súa
obra divulgativa \textit{A evolución da física} (1938), a gran motivación da
física é de índole metafísica. <<\textit{A través de todos os esforzos, en cada
unha das dramáticas loitas entre as concepcións vellas e as novas, recoñécese o
eterno anhelo de comprender, a crenza sempre firme na harmonía do mundo, crenza
continuamente reforzada polo encontro de obstáculos sempre crecentes cara a súa
comprensión}>>.

\end{multicols}
