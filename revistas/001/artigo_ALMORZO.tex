\Titular%
{Física dun bo almorzo: Chulas perfectas e cafés rebeldes}
{Ánxel Costas Castro}
{divulgacion}
{A física aplicada á cocción e vibracións do café.}


\begin{multicols}{2}

Meus ben queridos lectores, seguro que para moitos dos que estades lendo este
artigo o almorzo sexa unha das comidas máis gozadas do día (ou, polo menos, a
que necesita denantes de ser persoa) e seguramente non queredes un almorzo
ordinario, senón que queredes o mellor posíbel. Permitídeme, pois, axudarvos
nesta tarefa, a priori sinxela, pero que agocha máis física da que pensades. É
lóxico pensar se existe unha fórmula secreta para conseguir noso obxectivo. Por
desgraza, esta fórmula non existe, xa que cada un ten particularidades e gostos
de seu que fan que non exista o almorzo ideal único para todos, pero podemos
axudar a obter, por exemplo, unhas chulas perfectas.

Para iso teremos que saber que factores afectan á cociña deste rico alimento e
así, cada un poida axustar estas variábeis aos seus gostos. Nun acto tan
sinxelo como este, actúan a mecánica de fluídos, a termodinámica e a dinámica.

Para comezar, podemos considerar (creo que nisto estaremos tanto autor como
lector de acordo) que unha chula excelente ten unha forma e grosor uniforme, e
é esponxosa e dourada por fóra ao cocerse homoxeneamente. Para conseguir o
primeiro destes requisitos necesitamos esparcir a masa da chula adecuadamente
na tixola, pero se a masa é moi líquida esparcirase moi rápido e a chula
quedará demasiado delgada e, no caso de que sexa moi espesa, apenas poderá
moverse e quedará irregular. Para conseguir o equilibrio entre estas dúas
situacións teremos que traballar coa viscosidade, que, como todo físico que se
aprecie sabe (seguro que ti tamén o es), é a resistencia dun fluído a fluír. A
proporción dos nosos ingredientes cambiarán esta propiedade, xa que se temos
máis leite ou auga a viscosidade se reducirá, pero non é este o único factor,
pois a presenza de burbullas cando batimos demasiado cambia a fluidez, ou
mesmamente se aumentamos a temperatura, a viscosidade diminuirá.

Unha vez que consigamos a viscosidade desexada virá a cocción da chula, onde a
trasnferencia de calor principalmente será por conducción e convección. Na
conducción, o calor da tixola pasa á masa en contacto e vai determinar como de
rápido se forma a corteza dourada. Por outro lado, a convección será o calor
que vai dende a base da masa até a superficie superior; se non temos unha boa
conducción, poderíamos queimar a base antes que o interior estea ben feito.
Nestes procesos térmicos, a temperatura da tixola é chave. Unha temperatura
óptima sería entre 175 e 200 º$C$, ademais precisamos que a temperatura sexa
uniforme e non existan focos fríos e quentes (científicos con moito tempo libre
viron que se a temperatura da tixola é uniforme a probabilidade de queimar as
chulas caía nun 50 \%). Un bo material pode ser o ferro fundido, pois o teflón
pode quentar de xeito desigual ou unha tixola de aluminio pode perder
temperatura facilmente.

\begin{center}
    \includegraphics*[width=0.7\linewidth]{revistas/001/imaxes/cafeconleche.jpeg}
    \captionof{figure}{\small Mancha dun cafe derramado.}
\end{center}

Agora que xa temos a masa ideal e a temperatura e material da tixola máis
óptimas, chegamos a un punto crítico da nosa receta. Darlle a volta á masa (non
entres en pánico). Se es unha persoa precavida seguramente queiras axudarte de
utensilios de cociña, mais se queres impresionar a alguén, ou se simplemente
queres asemellarte a un gran chef, é probábel que queiras darlle a volta cun
golpe de pulso. Aquí hai que ter en conta a rotación e o momento angular,
ademais de controlar a nosa forza, xa que se nos pasamos, a chula xirará
demasiado rápido e caerá fóra. A aceleración angular máis óptima é de 5
$rad/s^2$, pero como isto é moi difícil de saber a ollo, podo dicirche que a
masa seguirá un movemento parabólico, para que, se te pasas ou quedas curto de
forza, saibas onde vai caer. Unha vez consigas dominar todo este proceso, xa
poderás degustar unhas chulas para lamber os dedos.

Por suposto, este delicioso almorzo non estaría completo sen o líquido
fundamental para o funcionamento correcto dun físico: o café. Mais hai que ter
coidado, porque un paso en falso, literalmente, podería arruinar o noso almorzo
ideal. Seguro que en máis dunha ocasión, mentres levávades o voso café ou volo
traían á vosa mesa da cafetaría, este acababa derramándose polos bordos da
taza e seguramente máis dunha vez pensástedes <<\textit{que patoso son}>> ou
<<\textit{non pode ser tan complicado levar o café sen derramalo}>>. Permíteme
dicirche a ti, meu ben querido lector, que teño unha notica boa e outra mala:
non es unha persoa patosa pero si que é un tema máis complicado do que pensas.

Normalmente o café derrámase aos 4 ou 5 metros (uns 7 ou 10 pasos), aínda que
poden ser máis se estás atento cando o leves. O motivo detrás deste indesexábel
fenómeno ten que ver coas oscilacións e, sendo máis precisos, coas resonancias,
mais acouga, voucho explicar sinxelamente para que comprendas os motivos de tal
desgraza e poidas evitalo, se es capaz diso claro está. Cando comezamos a
camiñar coa taza, a aceleración dos nosos pasos induce unha amplitude inicial
ao vaivén do café, sendo esta maior canto maior sexa a aceleración. Dito
noutras palabras, é oportuno que comeces a andar a miúdo para que partamos con
marxe e poder chegar até nosa mesa con éxito. Neste punto, se prestas atención
ao movemento do café, verás que a amplitude da oscilación comezará a aumentar e
seguramente non saibas por que; trátase dos nosos pasos. Máis ben do ruído que
xeran os nosos pasos, xa que este contén harmónicos de altas frecuencias que
convirten a taza co café nun sistema oscilatorio asimétrico inestábel. Como,
por desgraza, non podemos voar, teremos que coformarnos con camiñar
concentrados e a modiño para minizar o ruído que poidamos xerar ao camiñar e así
aumentar o tempo antes de que suceda a traxedia que tentamos evitar.

Os líquidos (os nosos preciados cafés) dentro de recipentes cilíndricos (súas
pequerrechas tazas) teñen unhas curvas de resonancia relativamente anchas, por
iso se excitan sinxelamente a pesares de estar lonxe da frecuencia de resonancia
(sempre que se supere un certo umbral de ruído). Este tedioso problema levou
aos físicos a deseñar tazas especiais que foran capaces de evitar o
desafortunado derrame, como tazas flexíbeis que amortiguan o vaivén e eliminan
as oscilacións, ou poñendo aneis concéntricos na parede interior da taza para
eliminar o fluxo de masa asociado á frecuencia de resonancia. Todos estes
modelos quedan no esquecemento xa que, ben sexa por simplicidade e costume ou
ben pola preguiza dos nosos, a forma dunha taza común segue sendo a dominante
hoxe en día, a pesar do seu efecto secundario: derramar o noso café. Con todo
isto, non nos queda máis remedio que andar con coidado e evitando facer
aceleracións moi fortes.

Se con isto que acabas de aprender, aínda así tes a mala sorte, a desgraza ou,
se cabe, a torpeza como para verter este líquido vital para nós, non te
enfades, xa que incluso a partir deste molesto accidente podemos aprender algo
novo antes de comezar a traballar. A gota derramada ou, para ser máis precisos,
a mancha que queda ao secarse garda un gran segredo. Se tes a paciencia e o
tempo, verás que a mancha que deixa o noso pequeno desastre comezará a ter unha
forma peculiar, e é que veremos que esta adopta unha forma de circunferencia en
vez dun circulo completo (supoñemos unha mancha circular, xa sabemos como nos
gosta aos físicos aproximar a casos sinxelos).

A que se debe esta rareza? Pois, cando a pinga de café cae enriba da mesa, os
bordes da mancha son máis estreitos ca o centro, polo que se evaporan máis
rápido. Como na evaporación pérdese líquido, aparece un fluxo capilar para
compensar esta perda, o cal arrastra consigo tamén ás partículas de café que
acaban acumulándose nos extremos. Deste xeito temos o patrón de aneis que tanto
nos chamou a atención. Este fenómeno coñécese como \textit{mancha de café} ou
efecto \textit{coffee ring}, e pode darse tamén (ademais de no café) en tinta,
sangue ou líquidos que conteñan partículas en suspensión.

Sabendo isto, se o lector é unha persoa astuta, intuirá que isto pode ter
aplicacións prácticas, e así é. No campo da medicina, sabendo como se evaporan
as gotas de sangue ou outros fluídos biolóxicos, poden crear un diagnóstico; no
campo dos materiais e da nanotecnoloxía, estúdase este fenómeno para mellorar a
distribución de nanopartículas sobre superficies co obxectivo de construír
sensores ou dispositivos electrónicos, como por exemplo paneis solares. Por
suposto, hai moitísimos máis campos e aplicacións para este curioso efecto.
Poucas veces estudar unha mancha resultou tan interesante e tan rentábel!

E, agora si, este querido lector xa poderá tomar o seu almorzo tranquilo, coa
seguridade de que deprendeu a cociñar as chulas ideais, como evitar derramar o
seu importantísimo café matinal e incluso como aproveitar unha mancha no noso
beneficio. Quizais por todo isto, e por máis, dise que o almorzo é a comida
máis importante do día. Que aproveite!

\end{multicols}
