\Titular%
{O Teorema de Bohr-Van Leeuwen: o segredo está na cuántica}%
{Luis Arcas Morcillo}%
{divulgacion}%
{Da imposibilidade clásica do magnetismo.}%


\begin{refsection}
\begin{multicols}{2}


\textit{Classical magnetism is fake}. Estou seguro de con este artigo a
polémica está servida. Gústame imaxinar que os nosos prezados lectores
dividiranse en dous bandos. Uns entrarán en pánico e ante a afirmación anterior
quererán abolir o estudo da magnetostática clásica en tódalas facultades do
país. Outros, quizais máis cínicos (e con razón), pensarán que ten moito
sentido que o magnetismo é unha consecuencia cuántica. A fin de contas, todo
fenómeno a escala atómica debe rexer as leis da mecánica cuántica. Non
obstante, este feito non era para nada coñecido no comezo do século XX, onde
comeza a nosa historia. Abrochade os vosos cintos, porque avecíñanse curvas.

\subsection*{O magnetismo previo ao átomo}

Supoño que os inicios da teoría do magnetismo son coñecidos pola maioría, pero
en resumidas contas aconteceu máis ou menos así:

Dende tempos inmemoriais, antigas civilizacións xa coñecían a ferrita, un
mineral que posuía a fascinante propiedade de atraer metais. Séculos despois,
científicos renomados como Faraday, Lenz e Maxwell contribuíron ao primeiro
formalismo dunha teoría que actualmente coñecemos como electromagnetismo (EM),
nunha formulación de campos que é de moita utilidade para sentar as bases da
electrodinámica. Así, o ser humano era agora capaz de simular as propiedades da
ferrita a partir de cargas eléctricas en movemento nun circuíto, mostrando o
vínculo profundo entre a electricidade e os imáns. No entanto, quedaba pendente
unha cuestión: por que hai materiais que presentan campos magnéticos propios
permanentes? Que propiedades ten a ferrita e outros materiais presentes na
natureza para actuar como imáns naturais?

O descubrimento do átomo e das primeiras partículas subatómicas tivo unha
grande relevancia non só a nivel experimental, senón tamén conceptual. Os
electróns e protóns teñen carga igual e de signo oposto, pola súa contra, as
masas son distintas, onde claramente é maior a masa do protón. Entón, ten
sentido pensar que estas partículas cargadas en particular, e os átomos en
xeral, son susceptibles individualmente á actuación de campos
electromagnéticos. A enerxía adquirida pola presenza dun campo magnético pode
describirse a partir do momento magnético da partícula cargada: 
\begin{equation}\label{ec.: energia_momento}
    \vec \mu = \frac{1}{2m}(\vec r\times q\vec v)\Longrightarrow E =
    -\vec \mu\vec B = -\mu_zB.
\end{equation}

O signo negativo corresponde ao feito de que a enerxía minimízase cando o
momento magnético $\vec \mu$ está aliñado co campo: $E=E_\mathrm{mín}
\Leftrightarrow \vec \mu \parallel \vec B \Leftrightarrow \vec \mu=\mu \hat z$. 

Sen entrar en moitos detalles, o comportamento dos momentos magnéticos dos
átomos individuais fronte a campos magnéticos permitiríanos describir o
material. Se tódolos momentos están aliñados entre eles, falamos de
\emph{ferromagnetismo}: o material presentaría un comportamento de imán natural
cando se introduce un campo magnético. Se os momentos magnéticos individuais
están desaliñados, de tal xeito que aparentemente opóñense ao campo externo,
falamos de \emph{diamagnetismo}. Sen embargo, se a interacción parece favorable
á dirección do campo, falaremos de \emph{paramagnetismo}.

\subsection*{Langevin e o primeiro modelo atómico do paramagnetismo}

En 1905, o físico francés Paul Langevin foi o primeiro en presentar un modelo
que fose capaz de explicar a escala atómica o comportamento dos distintos
materiais fronte a campos magnéticos. Actualmente estúdase como a \emph{teoría
semiclásica do paramagnetismo} \cite[Sec. 2.1.4.]{blundell.s_2001}. Langevin
propoñía que os átomos tiñan un momento magnético constante apuntando a certa
rexión do espazo segundo a distribución de Boltzmann:

\begin{equation}\label{ec.: bvl_boltzmann}
    dP = \frac{1}{Z}e^{\frac{E}{k_B T}}d\Omega,\,\, Z
    =\int_{0}^{2\pi}\int_{0}^{\pi}e^{\frac{E}{k_B T}}d\Omega,
\end{equation}

onde $d\Omega = \sin(\theta)d\theta d\phi$ é a diferencial do ángulo sólido.
Así, dados $N$ átomos dun material lineal nun volume $V$, a magnetización
resultante do material é a siguiente:

\begin{equation}\label{ec.: blv_langevin}
    M = \frac{N}{V} \int\mu_zdP = \frac{N}{V}\cdot
    L\left(\frac{B}{k_B T}\right),
\end{equation}

onde $L(y) = \coth(y)-y^{-1}$ é a función de Langevin. Dous anos máis tarde, o
tamén francés Pierre Weiss propuxo outro modelo que correxía o de Langevin e
introducía o comportamento ferromagnético de diversos materiais na teoría.

\subsection*{As disertacións de Bohr e Van Leeuwen. Unha nova teoría.}

Os problemas non tardarían en chegar. Xa por 1911, un novo Niels Bohr puxo en
dúbida a teoría de Langevin, achando unha contradición infranqueable se
soamente se empregaban técnicas de Mecánica Estadística. Anos máis tarde, unha
estudante da Universidade de Leiden, chamada Hendrika Johanna van Leeuwen
disertaría sobre o mesmo tema na súa tese doutoral, chegando ás mesmas
conclusións que Bohr. Vexamos cal foi o seu razoamento e as conclusións finais
\cite[Sec. 1.2.2.]{blundell.s_2001}.\\

Consideremos unha mostra paramagnética de $N$ átomos, todos coa mesma masa, no
seo dun campo magnético. Consideraremos exclusivamente o momento debido aos
electróns, sendo o resultado xeral para outras partículas cargadas. Tense entón
que cada electrón ten un momento magnético:
$$\vec \mu = \frac{-e}{2m} ({\vec r}\times \dot {\vec r}) \Rightarrow
\mu_z=\sum_{i=1}^{3N}a_i(q_1,\dots,q_{3N})\dot q_i.$$

É dicir, o momento magnético total nunha dirección (a dirección do campo $\vec
B$) é unha función explícita das velocidades nas coordenadas xerais.

Consideremos ademais que o noso sistema conserva a súa enerxía, de tal feito
que o total correspóndese co hamiltoniano, función de coordenadas xerais, $\vec
q$ e os seus momentos canónicos conxugados, $\vec p$. Para un sistema sometido
a un campo electromagnético caracterizado polos potenciais $(\phi,\vec A)$,
dáse o seguinte hamiltoniano :
\begin{equation}\label{ec.: bvl_hamiltoniano}
    E=H(\vec q,\vec p) = \frac{1}{2m}\sum_{i=1}^N (\vec p_i - e\vec
    A_i)^2 + e\phi(q_1,...,q_{3N})
\end{equation}

Podemos promediar os seus momentos magnéticos para achar a enerxía promedio do
sistema $\braket{E} = -B\braket{\mu_z}$. Segundo a Mecánica Estadística
\eqref{ec.: bvl_boltzmann} e o modelo de Langevin \eqref{ec.: blv_langevin},
podemos calcular este promedio a partir da función de partición $Z$ da mostra:
\begin{equation}\label{ec.: bvl-mec_est}
    \braket{\mu_z} = \int \mu_z\, dP = \frac{1}{Z}\int e^{-\beta
    H(\vec q,\vec p)} \, d^N(\vec q, \vec p),
\end{equation}

onde $\beta^{-1} = k_B T$. Recordando as relacións de conxugación canónica no
formalismo hamiltoniano, temos que $\dot q_i = \frac{\partial H}{\partial
p_i}$, e polo tanto o promedio cumpre que:
\begin{equation}
    \braket{\mu_z} = \frac{1}{Z}\iint\sum_{i=1}^{3N} a_i
    (q_1,...,q_{3N}) \frac{\partial H}{\partial p_i}e^{-\beta
    H}d^{3N}(\vec q,\vec p).
\end{equation}

No entanto, se integrarmos na totalidade do espazo de fases, veremos que o
promedio é igual a cero\footnote{É un bo exercicio para o lector facer os
cálculos, aquí vai unha pista: separe a integral múltiple no producto das
integrais na posición $\vec q$ e o seu momento $\vec p$. O integrando
dependente de $\vec p$ é impar, e polo tanto a integral anúlase.}. Deste modo,
baixo a acción do campo, en promedio estadístico hai tantos momentos aliñados
coma desaliñados: a mostra é diamagnética.\\

En resumidas contas, de ser pola teoría clásica, non habería magnetismo
posible, nunha clara contradición coa realidade experimental. Así, era
necesaria unha nova teoría que fose capaz de explicar a presenza do magnetismo
na natureza. Segundo \citet[pp. 354-355]{van-vleck.jh_1977}, foi este feito o
que motivou a Bohr a introducir a cuantización do momento angular orbital do
electrón ao redor do núcleo de hidróxeno para plantexar o seu famoso modelo
atómico. Así, un pode afirmar que o nacemento da cuántica está moi vinculado ao
magnetismo atómico. 

\subsection*{Conclusións}

En definitiva, aínda que a teoría de Maxwell é moi útil para explicar fenómenos
electromagnéticos clásicos coma a indución magnética ou mesmo a luz (o que
podería dar para outro artigo), a imposibilidade de explicar o fenómeno do
paramagnetismo pon de manifesto que fai falla unha teoría máis completa, que
teña presente o comportamento propio dos átomos. É dicir, que, unha vez máis, a
cuántica está aquí para salvarnos da catástrofe. Un desenvolvemento máis
completo do rol da cuántica no magnetismo pode cursarse na asignatura de
Nanomagnetismo e Nanotecnoloxía desta facultade.

\printbibliography

\end{multicols}
\end{refsection}
