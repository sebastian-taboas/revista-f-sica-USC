\Titular*%
{Pasatempos}%
{Luis Arcas, Víctor Díaz Díaz}%
{pasatempos}%
{}%
\vspace*{-6mm}
\section*{\textcolor{Resalte}{Encrucillado \#1}, Luis Arcas}

%{\CorResalte Encrucillado}
% Descomentar para entrar no modo solución
% No modo solución todos os puzzles saen resoltos
% \PuzzleSolution
\begin{Puzzle}{27}{16}
|[1]P  |{}    |{}    |{}    |{}    |{}    |{}    |{}    |{}    |{}    |{}    |{}    |{}    |{}    |{}    |{}    |{}    |[2]P  |{}    |{}    |{}    |{}    |{}    |{}    |{}    |{}    |{}   |{} |.
|É     |{}    |{}    |{}    |{}    |{}    |{}    |{}    |{}    |{}    |{}    |{}    |{}    |{}    |{}    |{}    |{}    |O     |{}    |{}    |[3]C  |{}    |{}    |{}    |{}    |[4]F  |{}   |{} |.
|N     |{}    |{}    |[5]L  |U     |[6]Z  |{}    |{}    |{}    |{}    |{}    |{}    |{}    |{}    |{}    |{}    |{}    |[7]I  |N     |T     |E     |N     |S     |I     |D     |A     |D    |E  |.
|D     |{}    |{}    |{}    |{}    |E     |{}    |{}    |{}    |{}    |[8]C  |{}    |{}    |{}    |{}    |{}    |{}    |S     |{}    |{}    |R     |{}    |{}    |{}    |{}    |S     |{}   |{} |.
|U     |{}    |{}    |[9]É  |T     |E     |R     |{}    |{}    |[10]G |R     |I     |F     |[11]F |I     |T     |H     |S     |{}    |{}    |N     |{}    |{}    |{}    |{}    |E     |{}   |{} |.
|L     |{}    |{}    |{}    |{}    |M     |{}    |{}    |{}    |{}    |I     |{}    |{}    |Í     |{}    |{}    |{}    |O     |{}    |{}    |{}    |{}    |{}    |{}    |{}    |S     |{}   |{} |.
|[12]O |S     |[13]C |I     |L     |A     |D     |O     |R     |E     |S     |{}    |{}    |S     |{}    |{}    |{}    |[14]N |E     |W     |T     |O     |N     |{}    |{}    |{}    |{}   |{} |.
|{}    |{}    |U     |{}    |{}    |N     |{}    |{}    |{}    |{}    |T     |{}    |{}    |I     |{}    |{}    |{}    |{}    |{}    |{}    |{}    |{}    |{}    |{}    |{}    |{}    |{}   |{} |.
|{}    |{}    |A     |{}    |{}    |{}    |{}    |{}    |{}    |{}    |A     |{}    |{}    |C     |{}    |{}    |{}    |{}    |{}    |{}    |{}    |{}    |{}    |{}    |{}    |{}    |{}   |{} |.
|[15]V |E     |N     |U     |S     |{}    |{}    |{}    |[16]V |O     |L     |F     |R     |A     |[17]M |I     |O     |{}    |{}    |[18]C |{}    |{}    |{}    |{}    |{}    |{}    |{}   |{} |.
|{}    |{}    |T     |{}    |{}    |{}    |{}    |{}    |{}    |{}    |{}    |{}    |{}    |{}    |O     |{}    |{}    |{}    |{}    |A     |{}    |{}    |{}    |{}    |{}    |{}    |{}   |{} |.
|{}    |{}    |O     |{}    |{}    |{}    |{}    |{}    |{}    |{}    |{}    |{}    |{}    |[19]U |M     |A     |{}    |{}    |{}    |U     |{}    |{}    |{}    |{}    |{}    |{}    |{}   |{} |.
|{}    |{}    |{}    |{}    |{}    |{}    |{}    |{}    |{}    |{}    |{}    |{}    |{}    |{}    |E     |{}    |{}    |{}    |{}    |S     |{}    |{}    |{}    |{}    |{}    |{}    |{}   |{} |.
|{}    |{}    |{}    |{}    |{}    |{}    |{}    |{}    |{}    |{}    |{}    |{}    |{}    |[20]I |N     |E     |R     |C     |I     |A     |{}    |{}    |{}    |{}    |{}    |{}    |{}   |{} |.
|{}    |{}    |{}    |{}    |{}    |{}    |{}    |{}    |{}    |{}    |{}    |{}    |{}    |{}    |T     |{}    |{}    |{}    |{}    |{}    |{}    |{}    |{}    |{}    |{}    |{}    |{}   |{} |.
|{}    |{}    |{}    |{}    |{}    |{}    |{}    |{}    |{}    |{}    |{}    |{}    |{}    |{}    |O     |{}    |{}    |{}    |{}    |{}    |{}    |{}    |{}    |{}    |{}    |{}    |{}   |{} |.
\end{Puzzle}

\begin{PuzzleClues}{\textbf{Vertical}}
\Clue{1}{PÉNDULO}{O legado de Foucault na facultade}
\Clue{2}{POISSON}{Probablemente un peixe manchado}
\Clue{3}{CERN}{Centro de investigación europeo máis internacional}
\Clue{4}{FASES}{Son lunares ou diagramas}
\Clue{6}{ZEEMAN}{Ver dobre en cuántica débese a este efecto}
\Clue{8}{CRISTAL}{Disposición atómica ordenada}
\Clue{11}{FÍSICA}{Rama científica especializada no estudo da natureza e comportamento da materia, así como a forza e a enerxía}
\Clue{13}{CUANTO}{Canta enerxía?}
\Clue{17}{MOMENTO}{Teno esta revista}
\Clue{18}{CAUSA}{Precede ao efecto non relativista}
\end{PuzzleClues}

\begin{PuzzleClues}{\textbf{Horizontal}}
\Clue{5}{LUZ}{Onda e corpúsculo}
\Clue{7}{INTENSIDADE}{Voltaxe = ... $\times$ Resistencia}
\Clue{9}{ÉTER}{Se non se sabe o que é, posiblemente é isto}
\Clue{10}{GRIFFITHS}{O gran libro (polo menos en EM)}
\Clue{12}{OSCILADORES}{En esencia, todo son ... (pl.)}
\Clue{14}{NEWTON}{Físico nacido en Natividad}
\Clue{15}{VENUS}{Afrodisíaco planeta}
\Clue{16}{VOLFRAMIO}{Elemento W}
\Clue{19}{UMA}{Unidad de masa atómica}
\Clue{20}{INERCIA}{Principio de ..., o primeiro de tres}
\end{PuzzleClues}

%Causa • CERN • Cristal • Cuanto • Fases • Física • Griffiths • Inercia •
%Intensidade • Luz • Momento • Newton • Osciladores • Poisson • Péndulo • UMA •
%Venus • Volframio • Zeeman • Éter

% Solucions cifradas con rot13 (movendo os caracteres 13 adiante) 
% Pnhfn PREA Pevfgny Phnagb Snfrf Sífvpn Tevssvguf Varepvn Vagrafvqnqr Yhm
% Zbzragb Arjgba Bfpvynqberf Cbvffba Céaqhyb HZN Irahf Ibysenzvb Mrrzna Égre

\section*{\textcolor{Resalte}{Colgando cadros}, Víctor Díaz Díaz
}

% Título
%Colgando cadros
%
% Autor
%Víctor Díaz Díaz

% Corpo
\begin{refsection}
\begin{multicols}{2}

\subsection*{Enunciado}

Queremos colgar un cadro dunha parede empregando $n$ cravos, de xeito que
quitando \textbf{un único cravo calquera} o cadro caia polo seu peso, ignorando
fricción. Para o caso $n=1$, a solución sería a mostrada na figura \textbf{AQUI
HAI QUE POÑER A FIGURA CA SOLUCION}
% \ref{fig:sol_n1}.

\begin{itemize}
    \item[$\bullet$] Nivel básico: $n=2$.
    \item[$\bullet$] Nivel avanzado: $n=3$.
    \item[$\bullet$] Nivel experto: $n$ arbitrario.
\end{itemize}


\subsection*{Criterios de éxito}

\begin{itemize}
    \item[$\bullet$] Nivel básico: solución gráfica ou solución formal.
    \item[$\bullet$] Nivel avanzado: solución gráfica ou solución formal.
    \item[$\bullet$] Nivel experto: solución formal.
\end{itemize}

Nótese que hai varias solucións posibles para o Nivel experto.

\subsection*{Solucións}

\subsubsection*{Solucións gráficas}

\begin{itemize}
    \item[$\bullet$] Nivel básico: véxase a figura 1(b) de \cite{Demaine_2013}.
    \item[$\bullet$] Nivel avanzado: véxase a figura 4(c) de \cite{Demaine_2013}.
    \item[$\bullet$] Nivel experto: ningunha.
\end{itemize}

\subsubsection*{Solución formal}

A solución ao problema fai uso de conceptos de teoría de grupos, concretamente
de grupos libres. Todo o que se vai discutir a continuación foi extraído de
\cite{Demaine_2013}.

Dado un conxunto de $n$ elementos $C(n) = {x_i / i=1,2,...,n}$, denominados
\textit{xeradores}, contrúese un \textit{grupo libre} $F_C = (C(n),\cdot)$ a
partir de todas as palabras que se poden formar operando os xeradores $x_i$ e
os seus inversos $x_i^{-1}$. Deste xeito tamén definimos a identidade como $e =
x_i \cdot x_i^{-1}, \forall i$ (en adiante omitiremos o símbolo da operación).

No noso caso, $n$ será igual ao número de cravos e cada elemento de $C(n)$
corresponderáse coa forma de facer pasar a corda ao redor do cravo: $x_i$
representa un xiro ao redor do cravo $i$-ésimo no sentido horario, $x_i^{-1}$
representa un xiro ao redor do cravo $i$-ésimo no sentido antihorario e a
identidade $e$ correspóndese con non facer ningún xiro. Así, podemos escribir
cada xeito de colgar o cadro cunha palabra de $F_C$ e o noso obxectivo e atopar
aquelas palabras para as que, eliminando os $x_i$ e $x_i^{-1}$ para un $i$
concreto, a palabra resultante se simplifique ata corresponderse coa
identidade.

Para $n=1$, as únicas posibles palabras son $S_1 = x_1$ ou $S'_1=x_1^{-1}$,
aínda que realmente $S'_1 = (S_1)^{-1}$. Esto vai pasar para cada solución que
atopemos, e débese ao feito de que $S_n$ e $S_n^{-1}$ se corresponden ca mesma
forma de colgar o cadro, simplemente cambiando o sentido no que se recorren os
cravos. Neste caso é evidente comprobar que eliminar $x_1$ nos deixa sen
palabra, e o cadro cae.

Para $n=2$, a cantidade de palabras que se poden construir aumenta. Sen
embargo, non é complicado comprobar por forza bruta que a palabra $S_2 = x_1
x_2 x_1^{-1} x_2^{-1}$ é a solución ao problema. De novo, eliminando o cravo
$i$-ésimo suprimimos os $x_i$ e $x_i^{-1}$ da palabra, o que produce $x_1
x_1^{-1} = x_2 x_2^{-1} = e$, e o cadro cae. De cara a xeralizar o resultado
para $n>2$, cómpre notar que a palabra $S_2$ se corresponde co
\textit{conmutador} de $x_1$ e $x_2$ (ollo, a pesar de chamarse igual a súa
definición en teoría de grupos é diferente á empregada en física cuántica).
Teremos entón $S_2 = [x_1,x_2] = x_1 x_2 x_1^{-1} x_2^{-1}$.

Para $n>2$, a solución constrúese de xeito recursivo a partir da solución para
$n-1$ cravos. Nótese que se cumple $S_2 = [S_1,x_2]$, polo que a solución xeral
será

\begin{equation}
    S_n = [S_{n-1},x_n] = S_{n-1} x_n S_{n-1}^{-1} x_n^{-1},
    ~~ \forall n \geq 2,
\end{equation}

\noindent onde se empregan as relacións
alxebraicas $(x y)^{-1} = y^{-1} x^{-1}$ e $(x^{-1})^{-1} = x$.

Esta construción é válida para calquera $n$. A lonxitude das palabras que son
solución ao problema medra de xeito exponencial, tendo concretamente $2^n +
2^{n-1} - 2$ símbolos na palabra $S_n$. Sen embargo, non é o xeito máis
eficiente de construir solucións, e pode facerse tamén de xeito polinómico.

\cite{Demaine_2013}
\printbibliography

\end{multicols}
\end{refsection}